\documentclass[dvipdfmx, 11pt, notheorems]{beamer}

\usepackage{bxdpx-beamer}
\usepackage{pxjahyper}
\usepackage{amsmath, amssymb, amsthm}
\usepackage{bm}
\usepackage{stmaryrd}
\usepackage{aliascnt}
\usepackage{tikz}
\usepackage{chngcntr}
\usepackage[capitalize]{cleveref}
\usepackage{appendixnumberbeamer}

\renewcommand{\kanjifamilydefault}{\mcdefault}
\renewcommand{\kanjiseriesdefault}{m}

\newcommand{\tsum}{\mathrm{sum}}
\newcommand{\tadj}{\mathrm{adj}}
\newcommand{\aut}{\mathrm{Aut}}
\newcommand{\isom}{\xrightarrow{\sim}}
\newcommand{\mor}{\mathrm{Mor}}
\newcommand{\rank}{\mathrm{rank}}
\newcommand{\sign}{\mathrm{sign}}
\newcommand{\tor}{\mathrm{Tor}}
\newcommand{\Ker}{\mathrm{Ker}}
\newcommand{\Hom}{\mathrm{Hom}}
\newcommand{\im}{\mathrm{Im}}
\newcommand{\Span}{\mathrm{span}}
\newcommand{\ZZ}{\mathbb{Z}}
\newcommand{\QQ}{\mathbb{Q}}
\newcommand{\NN}{\mathbb{N}}
\newcommand{\RR}{\mathbb{R}}
\newcommand{\CC}{\mathcal{C}}
\newcommand{\VV}{\mathcal{V}}
\newcommand{\calA}{\mathcal{A}}
\newcommand{\calB}{\mathcal{B}}
\newcommand{\ob}{\mathcal{O}\mathrm{b}}
\newcommand{\graphimg}[2][]{
  \includegraphics[#1]{Tikz/Graphs/out/#2}
}

\newcommand{\commutimg}[2][]{
  \includegraphics[#1]{Tikz/Commutative/out/#2}
}
\newcommand{\figcommu}[2][]{
  \begin{center}
    \commutimg[#1]{#2}
  \end{center}
}

\newcommand{\boxprod}{\mathbin{\square}}




\usetheme{AnnArbor}
\usecolortheme{crane}
\usefonttheme{professionalfonts}

\setbeamertemplate{navigation symbols}{}
\setbeamertemplate{theorems}[numbered]

\theoremstyle{definition}
\newtheorem{definition}{Definition}
\newtheorem{theorem}{Theorem}
\newtheorem{lemma}{Lemma}
\newtheorem{example}{Example}
\theoremstyle{plain}
\newtheorem{proposition}{Proposition}
\newtheorem{corollary}{Corollary}

\title[Magnitude Homology]{The magnitude of graphs and the magnitude homology as its categorification}
\subtitle{2026年度 卒業研究発表}
\author[Kensho Yachi]{谷内 賢翔(やち けんしょう)}
\institute[YNU]{横浜国立大学 理工学部 数理科学EP}
\date{2026年2月17日}

\begin{document}

\setlength\abovedisplayskip{3pt}
\setlength\belowdisplayskip{3pt}
\setlength\abovedisplayshortskip{-8pt}
\setlength\belowdisplayshortskip{2pt}

\begin{frame}
  \titlepage
\end{frame}

\begin{frame}{目次}
  \tableofcontents
\end{frame}

\section{前提知識 1}

\begin{frame}{グラフと距離(前提)}
  \begin{itemize}
    \item グラフ $\rightarrow$ ループと多重辺を許さない有限グラフ
    \item グラフ $G$ の頂点集合を $V(G)$, 辺集合を $E(G)$ と表す
    \item \colorbox{yellow!30}{\parbox{0.9\linewidth}{
      グラフ $G$ が与えられたとき,頂点集合 $V(G)$ に,最短経路が定める距離 $d$ を与え,グラフ自体を(一般化)距離空間とみなす.
    }}
    
    一辺の長さを1とする.
    \item 上記で定義した距離は,非負整数および $\infty$ の値をとる
    \item 頂点が連結されていないとき,距離が $\infty$ となる
  \end{itemize}
\end{frame}

\begin{frame}{グラフのMagnitude}
  
  \begin{definition}[Leinster, 2014]
    グラフ $G$ の \textbf{similarity matrix} $Z_G (q)$を, その成分が $Z_G(q)(x,y) = q^{d(x,y)}$ である $|V(G)| \times |V(G)|$-行列として定義する.
  \end{definition}
  ここで,$q$ は変数であり,$q^{\infty} = 0$ とする.

  このとき,$Z_G(q)$ の対角成分は全て $1$ なので,$\det Z_G$ の定数項は $1$ であり, よって $Z_G$ は可逆である.
  \begin{definition}[Leinster, 2014]\label{magnitude}
    グラフ $G$ の \textbf{magnitude} は次のように定義される:
    \[
      \# G(q) = \tsum(Z_G(q)^{-1}) \in \mathbb{Q}(q).
    \]
    (逆行列の全成分の和)
  \end{definition}
  ここで, $\tsum$ は行列の全成分の和を表す.
\end{frame}

\begin{frame}{例: 完全グラフ $K_3$}
  \begin{columns}
    \begin{column}{0.4\textwidth}
      \centering
      \begin{tikzpicture}
        \foreach \i in {1,2,3}
          \coordinate (v\i) at (90+120*\i:1.2);
        \draw (v1) -- (v2) -- (v3) -- cycle;
        \foreach \i in {1,2,3}
          \fill (v\i) circle (2pt);
      \end{tikzpicture}
      \\ 完全グラフ $K_3$
    \end{column}
    \begin{column}{0.6\textwidth}
      $K_3$ のsimilarity matrixは次のようになる:
      \[
        Z_{K_3} = \begin{pmatrix} 1 & q & q \\ q & 1 & q \\ q & q & 1 \end{pmatrix}.
      \]
      
    \end{column}
  \end{columns}
  \vspace{0.5em}
  その逆行列 $Z_{K_3}^{-1}$ は
  \[
    Z_{K_3}^{-1} = \frac{1}{(1+2q)(q-1)} \begin{pmatrix} -q-1 & q & q \\ q & -q-1 & q \\ q & q & -q-1 \end{pmatrix}.
  \]

  よって, その成分和は
      \[
        \# K_3(q) = \frac{3}{1+2q}.
      \]

  \vspace{0.5em}
  一般に, \[\# K_n (q) = \frac{n}{1 + (n-1)q} \quad \text{(後ほど証明)}.\]
\end{frame}

\begin{frame}{Magnitudeを $\ZZ \llbracket q \rrbracket$ の元として見る}
  定義より, 
  \[
    \# G(q) = \tsum(Z_G(q)^{-1}) = \frac{\tsum(\tadj(Z_G(q)))}{\det(Z_G(q))}
  \]
  である.ここで,$\tsum$ は行列の全成分の和, $\tadj$ は余因子行列を表す.

  これと, $\det(Z_G)$ の定数項が $1$ であることより, $\# G(q)$ は $\ZZ \llbracket q \rrbracket$ に属する. 

  $\ZZ \llbracket q \rrbracket$ で見ることが今後重要となる.

  \vspace{1em}
  ※ $\ZZ \llbracket q \rrbracket = \{ \sum_{n=0}^\infty a_n q^n \mid a_n \in \ZZ \}$ (整係数形式的冪級数環)
\end{frame}

\begin{frame}{Weighting}
  \begin{definition}[Leinster, 2014]
    グラフ $G$ の各頂点 $x$ に対し, 
    \[
      w_G(x)(q) = \sum_{y \in V(G)}(Z_G(q)^{-1})(x,y)
    \]
    で関数 $w_G: V(G) \to \QQ (q)$ を定義し, これを $G$ の \textbf{weighting} と呼ぶ.
  \end{definition}
  定義から, $\# G (q) = \sum_{x \in V(G)} w_G(x)(q)$ が成り立つ.

  また, $G$ の各頂点 $x$ に対し, weighting $w_G(x)$ は次の方程式を満たす:
   \[
    \sum_{y \in V(G)} q^{d(x,y)} w_G(y) = 1 \quad \text{ for } x \in V(G).
  \]
  これを, \textbf{weighting equation} と呼ぶ. $Z_G(q)$ の可逆性より, weighting equation を満たす関数 $V(G) \to \QQ(q)$ は一意的である.
\end{frame}

\begin{frame}{実際に計算する}
  \begin{lemma}[Leinster, 2014]\label{vertex_transitive_magnitude}
    グラフ $G$ が vertex-transitive ならば, 任意の頂点 $x \in V(G)$ に対し,
    \[
    \# G(q) = \frac{|V(G)|}{\sum_{y \in V(G)} q^{d(x,y)}}.
    \]
  \end{lemma}
  ※ \textbf{vertex-transitive}: 作用 $\aut (G) \curvearrowright V(G)$ の軌道が唯1つであるグラフ.
\end{frame}

\begin{frame}{実際の計算例: 完全グラフ $K_n$}
  \begin{block}{Lemma 1 (再掲)}
    グラフ $G$ が vertex-transitive ならば, 任意の頂点 $x \in V(G)$ に対し,
    \[
    \# G(q) = \frac{|V(G)|}{S}.
    \]
    ただし, $S = \sum_{y \in V(G)} q^{d(x,y)}$.
  \end{block}
  $n$頂点完全グラフ $K_n$ は vertex-transitive である.
  Lemma 1 において, $S = 1 + (n-1)q$ であるから,
  \begin{align*}
    \# K_n(q) &= \frac{n}{1 + (n-1)q} \\
    &= n \sum_{k=0}^{\infty} - \{(n-1)q \}^k \\
  \end{align*}
  $k$ 次の係数は $c_k = \{ -(n-1)q \}^k n$ である. 特に, $c_0 = n, c_1 = -(n-1)n$.
\end{frame}

\begin{frame}{実際の計算例2: Petersenグラフ}
  \begin{block}{Lemma 1 (再掲)}
    グラフ $G$ が vertex-transitive ならば, 任意の頂点 $x \in V(G)$ に対し,
    \[
    \# G(q) = \frac{|V(G)|}{S}.
    \]
    ただし, $S = \sum_{y \in V(G)} q^{d(x,y)}$.
  \end{block}
  \begin{columns}[c]
    \begin{column}{0.35\textwidth}
      \centering
      \graphimg[width=\linewidth]{petersen.pdf}
    \end{column}
    \begin{column}{0.55\textwidth}
      $G$ を左図のグラフ (Petersen グラフ) とすると, $G$ もvertex-transitive となる (回転と内外の入れ替え). $S = 1 + 3q + 6q^2$ なので, 
      \begin{align*}
        \# G (q) &= \frac{10}{1 + 3q + 6q^2} \\
        &= 10 - 30q + \cdots.
      \end{align*}
      \bigskip
    \end{column}
  \end{columns}
\end{frame}



\begin{frame}{組み合わせ的表現}
  \begin{proposition}[Leinster, 2014]
    $G$ をグラフとするとき, 
    \[
    \begin{split}
      \# G(q) &= \sum_{k=0}^{\infty} (-1)^k \sum_{x_0 \neq \cdots \neq x_k} q^{d(x_0, x_1) + \dots + d(x_{k-1}, x_k)} \\
      &= \sum_{n=0}^{\infty} c_n q^n
    \end{split}
    \]
    が成り立つ.ここで, 
    \begin{multline*}
      c_n = \sum_{k=0}^{n} (-1)^k \Bigl| \Bigl\{ (x_0, \dots, x_k) \Bigm| x_0 \neq \cdots \neq x_k, \\
       d(x_0, x_1) + \dots + d(x_{k-1}, x_k) = n \Bigr\} \Bigr|.
    \end{multline*}
  \end{proposition}
\end{frame}



\begin{frame}{Proposition 1 からわかること}
  \begin{block}{Proposition 1の $c_n$ の式}
    \begin{multline*}
      c_n = \sum_{k=0}^{n} (-1)^k \Bigl| \Bigl\{ (x_0, \dots, x_k) \Bigm| x_0 \neq \cdots \neq x_k, \\
       d(x_0, x_1) + \dots + d(x_{k-1}, x_k) = n \Bigr\} \Bigr|.
    \end{multline*}
  \end{block}
    上記より,
  \begin{corollary}[Leinster \cite{Leinster2}, 2016]
    \[
      |V(G)| = c_0, \quad |E(G)| = -\frac{1}{2} c_1
    \]
  \end{corollary}
\end{frame}

\begin{frame}{Convexとprojection}
  \begin{definition}[Leinster, 2014]
    $X$ をグラフ, $U$ を $X$ の部分グラフとする.
    グラフ $X, U$ の距離をそれぞれ $d_X, d_U$ とする.

    $U$ が $X$ で \textbf{凸である} とは, 任意の頂点 $x,y \in V(U)$ に対し, $d_U(x,y) = d_X  (x,y)$ が成り立つことをいう.
  \end{definition}
  \begin{definition}[Leinster, 2014]
    $X$ をグラフ, $U$ を $X$ で凸な $X$ の部分グラフとする. また, 
    $V_U(X) = \{ v \in V(X) \mid d_X(v,u) < \infty \text{ for some } u \in V(U) \}$ とする
    
    ($U$ に接続されている頂点全体). 
    
    $X$ \textbf{projects to $U$} とは, 
    任意の $x \in V_U(X)$ に対し, ある $u' \in V(U)$ が存在して, 任意の $u \in V(U)$ について $d_X(x,u) = d_X(x, u') + d_X(u',u)$ が成り立つことをいう. 各 $x \in V_U(X)$ に対し, そのような $u'$ を1つ固定し, これを $\pi(x)$ と表す.
  \end{definition}
\end{frame}

\begin{frame}{projectの例}
  (赤に接続された)すべての頂点に対し,赤の中から1点選んでfactor throughすることで, 赤の全ての点との距離が最短になるようにできる.
  
  上記の $u'$ は一意に定まる.
  \begin{center}
    \graphimg[height=4cm, keepaspectratio]{K_mn_2.pdf}
    \hspace{1cm}
    \graphimg[height=4cm, keepaspectratio]{Grid_Graph_Red_U.pdf}
  \end{center}
\end{frame}

\begin{frame}{projectでない例}
  \begin{center}
    \graphimg[width=0.5\linewidth]{C_3_red.pdf}
  \end{center}
\end{frame}


\section{主結果 1}
\begin{frame}{主結果}
  \begin{theorem}[Leinster, 2014]
    $X$ をグラフ,  $G,H$ を $X$ の部分グラフとし, $X = G \cup H$ とする.
    $G \cap H$ が $X$ で凸であり, $H$ projects to $G \cap H$ ならば,
    \[
      \# X = \# G + \# H - \#(G \cap H).
    \]
  \end{theorem}

  上記の状況を満たす組 $(X;G,H)$ を, \textit{projecting decomposition} と呼ぶ.
  
  \begin{proof}
    $w_G + w_H - w_{G \cap H}$ が$X$ のweighting equationを満たすので, $w_X = w_G + w_H - w_{G \cap H}$ が導かれる.
    weighting equationを満たすことは, 計算により確認できる.
  \end{proof}
\end{frame}

\begin{frame}{計算の様子}
  \tiny
  \begin{align*}
    d(g, u) + d(u, h) &= d(g, u) + d(u, \pi(h)) + d(\pi(h), h) \\
    &\geq d(g, \pi(h)) + d(\pi(h), h) \\
    &\geq d(g, h) \\
    &= d(g, u) + d(u, h)
  \end{align*}
  means
  \[
    d(g, h) = d(g, \pi(h)) + d(\pi(h), h).
  \]
  If $x \in V(G)$, 
  \begin{align*}
    &\sum_{g \in V(G)}q^{d(g, x)} w_G(g) + \sum_{h \in V(H)} q^{d(h, x)} w_H(h) - \sum_{u \in V(G \cap H)} q^{d(u, x)} w_{G \cap H}(u) \\
    &= 1 + \sum_{h \in V(H)} q^{d(h, x)} w_H(h) - \sum_{u \in V(G \cap H)} q^{d(u, x)} \sum_{h \in \pi^{-1}(u)} q^{d(h, u)}w_H(h) \\
    &= 1 + \sum_{h \in V_{G \cap H}(H)} q^{d(h, x)} w_H(h) - \sum_{h \in V_{G \cap H}(H)} q^{d(x, \pi(h)) + d(\pi(h), h)} w_H(h) \\
    &= 1 + \sum_{h \in V_{G \cap H}(H)} q^{d(h, x)} w_H(h) - \sum_{h \in V_{G \cap H}(H)} q^{d(h, x)} w_H(h) \\
    &= 1.
  \end{align*}
  If $x \in V_{G\cap H}(H)$, 
  \begin{align*}
    &\sum_{g \in V(G)}q^{d(g, x)} w_G(g) + \sum_{h \in V(H)} q^{d(h, x)} w_H(h) - \sum_{u \in V(G \cap H)} q^{d(u, x)} w_{G \cap H}(u) \\
    &= \sum_{g \in V(G)}q^{d(g, \pi(x)) + d(\pi(x), x)} w_G(g) + 1 - \sum_{u \in V(G \cap H)} q^{d(u, \pi(x)) + d(\pi(x), x)} w_{G \cap H}(u)
  \end{align*}
\end{frame}

\begin{frame}{主結果1の使用例 : Bull Graph}
  \begin{block}{Theorem 1}
    \[
      \# X = \# G + \# H - \#(G \cap H).
    \]
  \end{block}
  \begin{columns}[c]
    \begin{column}{0.5\textwidth}
      \centering
      \begin{tikzpicture}[
        scale=0.8,
        dot/.style={circle, fill=black, inner sep=2pt, outer sep=1pt}, 
        thick
      ]
      \node[dot] (t_v1) at (0, 0) {};
      \node[dot] (t_v2) at (1, 0) {};
      \node[dot] (t_v3) at (0.5, -0.866) {};
      \node[dot] (t_v4) at (-0.5, 0.5) {};
      \node[dot] (t_v5) at (1.5, 0.5) {};
      \draw (t_v1) -- (t_v2) -- (t_v3) -- (t_v1);
      \draw (t_v1) -- (t_v4);
      \draw (t_v2) -- (t_v5);

      \draw[->] (0.5, -1.1) -- (0.5, -1.4);

      \begin{scope}[shift={(0, -2.3)}]
        \begin{scope}[shift={(-0.25, 0)}]
          \node[dot] (m1_v1) at (0, 0) {};
          \node[dot] (m1_v2) at (1, 0) {};
          \node[dot] (m1_v3) at (0.5, -0.866) {};
          \node[dot] (m1_v4) at (-0.5, 0.5) {};
          \draw (m1_v1) -- (m1_v2) -- (m1_v3) -- (m1_v1);
          \draw (m1_v1) -- (m1_v4);
        \end{scope}
          
        \begin{scope}[shift={(1.25, 0)}]
          \node[dot] (m2_v1) at (0, 0) {};
          \node[dot] (m2_v2) at (0.5, 0.5) {};
          \draw (m2_v1) -- (m2_v2);
        \end{scope}
      \end{scope}

      \draw[->] (0.5, -3.4) -- (0.5, -3.7);

      \begin{scope}[shift={(0, -4.6)}]
        \begin{scope}[shift={(-0.5, 0)}]
          \node[dot] (b1_v1) at (0, 0) {};
          \node[dot] (b1_v2) at (-0.5, 0.5) {};
          \draw (b1_v1) -- (b1_v2);
        \end{scope}
        
        \begin{scope}[shift={(0, 0)}]
          \node[dot] (b2_v1) at (0, 0) {};
          \node[dot] (b2_v2) at (1, 0) {};
          \node[dot] (b2_v3) at (0.5, -0.866) {};
          \draw (b2_v1) -- (b2_v2) -- (b2_v3) -- (b2_v1);
        \end{scope}
        
        \begin{scope}[shift={(1.5, 0)}]
          \node[dot] (b3_v1) at (0, 0) {};
          \node[dot] (b3_v2) at (0.5, 0.5) {};
          \draw (b3_v1) -- (b3_v2);
        \end{scope}
      \end{scope}
      \end{tikzpicture}
    \end{column}

    \begin{column}{0.5\textwidth}      
      \bigskip
      \vspace{-1em}
      左図のグラフ $G$ 場合, 
      $G = K_2 \vee C_3 \vee K_2$ とみなせる ($\vee$ は, 1点の同一視).
      この場合, 1点和は projecting decomposition を与えるので, 主定理を2回適応することにより, 以下のようになる.
      \begin{align*}
        \# G &= \# K_3 + 2 \cdot \# K_2 - 2 \cdot \# K_1 \\
            &= \frac{3}{1 + 2q} + 2 \cdot \frac{2}{1 + q} - 2 \cdot 1 \\
            &= 5 - 10q + \cdots.
      \end{align*}
    \end{column}
  \end{columns}
  次に, このMagnitudeという性質に, 代数トポロジー的な情報を増やすこと(categorification)を考える.
\end{frame}

\section{前提知識 2}

\begin{frame}{Magnitude complex}
  \begin{definition}[Hepworth, Willerton, 2017]
    グラフ $G$, 非負整数 $k,l$ に対し, 自由アーベル群
    \[
    \begin{split}
      MC_{k,l}(G) = \Span_{\ZZ} \bigl\{ (x_0, \dots, x_k) \bigm| x_i \in V(G), x_0 \neq \cdots \neq x_k, \\
      L(x_0, \dots, x_k) := d(x_0, x_1) + \cdots + d(x_{k-1}, x_k) = l \bigr\}
    \end{split}
    \]
    を定め, 境界準同型 $\partial : MC_{k,l}(G) \rightarrow MC_{k-1,l}(G)$ を
    \[
      \begin{split}
        &\partial = \sum_{i=1}^{k-1} (-1)^{i-1} \partial_i,~\partial_i (x_0, \dots, x_k)  \\
        &=\begin{cases}
          (x_0, \dots, \hat{x_i}, \dots, x_k) & \text{if } L(x_0, \dots, \hat{x_i}, \dots, x_k) = L(x_0, \dots, x_k), \\
          0 & \text{otherwise}.
        \end{cases}
      \end{split}
      \]
    で定義する.
  \end{definition}

  \vspace{0.6em}
  このように定義された $\partial$ は, 2回適用すると0になり, これにより各 $l$ に対し, 鎖複体 $(MC_{*,l}(G), \partial)$ が得られる.
  
\end{frame}

\begin{frame}{Magnitude homology}
  \begin{definition}[Hepworth, Willerton, 2017]
    グラフ $G$ の \textbf{magnitude homology} とは, homology
    \begin{align*}
      MH_{*, l}(G) &= H(MC_{*, l}(G), \partial) \\
      (MH_{k, l}(G) &= \Ker \partial \cap MC_{k, l}(G) / \im \partial \cap MC_{k, l}(G))
    \end{align*}
    のことである. 各 $l$ について得られる.
  \end{definition}
\end{frame}

\section{主結果 2}

\begin{frame}{Magniude との関係}
  \begin{theorem}[Hepworth, Willerton, 2017]
    $G$ をグラフとするとき, 
    \[
      \# G(q) = \sum_{k,l} (-1)^k \rank (MH_{k,l}(G)) q^l.
    \]
  \end{theorem}

  \vspace{-0.3em}

  \begin{proof}
    各 $k, l$ について,
    \[
      \rank (MC_{k,l}(G))q^l = \sum_{x_0 \neq \cdots \neq x_k} q^{d(x_0, x_1) + \cdots + d(x_{k-1}, x_k)}
    \]
    であること, 複体のオイラー標数に関する定理, Proposition 1 を利用.
  \end{proof}

  \vspace{-0.3em}

  \begin{block}{Proposition 1 (再掲)}
    \[
      \# G(q) = \sum_{k=0}^{\infty} (-1)^k \sum_{x_0 \neq \cdots \neq x_k} q^{d(x_0, x_1) + \dots + d(x_{k-1}, x_k)}.
    \]
  \end{block}
\end{frame}

\begin{frame}{Magnitude Homology の例}
  \begin{columns}
    \begin{column}{0.3\textwidth}
      \centering
      \graphimg[width=0.9\linewidth]{C_5.pdf}
    \end{column}
    \begin{column}{0.7\textwidth}
      各 $k, l$ についての $MH_{k,l}(C_5)$ のランク:
      \includegraphics[width=\linewidth]{Tables/out/mag_hom_C_5_v2.pdf}

    \end{column}
  \end{columns}
\end{frame}




\section{今後の課題}

\begin{frame}{今後の課題}

  \begin{tikzpicture}[
    scale=1.0,
    dot/.style={circle, fill=black, inner sep=2pt, outer sep=0pt},
    shared/.style={circle, fill=red, inner sep=2pt, outer sep=0pt},
    thick
    ]
  \begin{scope}[shift={(-3.5, 0)}]
    \node[shared, label=above:{$u$}] (u) at (0, 1) {};
    \node[shared, label=below:{$v$}] (v) at (0, -1) {};
    
    \node[dot] (l1) at (-1.5, 0) {};
    \node[dot] (l2) at (-1, 1.5) {};
    \draw (u) -- (l1) -- (v) -- (u);
    \draw (u) -- (l2);
    
    \node[dot, draw=blue, fill=blue] (r1) at (1, 1) {};
    \node[dot, draw=blue, fill=blue] (r2) at (1.5, 0) {};
    \node[dot, draw=blue, fill=blue] (r3) at (1, -1) {};
    \node[dot, draw=blue, fill=blue] (r4) at (2, 1) {};
    \draw[blue] (u) -- (r1) -- (r2) -- (r3) -- (v);
    \draw[blue] (r1) -- (r4);
  \end{scope}

  \draw[<->, very thick] (-1, 0) -- (1, 0);

  \begin{scope}[shift={(3.5, 0)}]
    \node[shared, label=above:{$u$}] (u) at (0, 1) {};
    \node[shared, label=below:{$v$}] (v) at (0, -1) {};
    
    \node[dot] (l1) at (-1.5, 0) {};
    \node[dot] (l2) at (-1, 1.5) {};
    \draw (u) -- (l1) -- (v) -- (u);
    \draw (u) -- (l2);
    
    \node[dot, draw=blue, fill=blue] (r1) at (1, -1) {};
    \node[dot, draw=blue, fill=blue] (r2) at (1.5, 0) {};
    \node[dot, draw=blue, fill=blue] (r3) at (1, 1) {};
    \node[dot, draw=blue, fill=blue] (r4) at (2, -1) {};
    \draw[blue] (v) -- (r1) -- (r2) -- (r3) -- (u);
    \draw[blue] (r1) -- (r4);
  \end{scope}
  \end{tikzpicture}

  \textbf{今後の課題:}
  \begin{itemize}
    \item \textbf{Whitney Twist:} ツイスト操作は Magnitude Homology を保存するか?
    \item \textbf{Diagonal Graphs} ($k \neq l$ で $MH_{k,l}=0$ となるグラフ) についての特徴付け
    \item \textbf{距離空間, enriched categories:} Magnitudeは, もともと (コンパクトでpositive definite な)距離空間, さらに広く, enriched categoriesに対して定義されている. 他に有用なクラスがあるか?
  \end{itemize}
\end{frame}

\begin{frame}{参考文献}
  \begin{thebibliography}{9}

  \bibitem{HepworthWillerton}
  Richard Hepworth and Simon Willerton.
  \textit{Categorifying the magnitude of a graph.}
  Homology Homotopy Appl. 19 (2017), no. 2, 31--60.

  \bibitem{Kelly}
  G. M. Kelly.
  \textit{Basic concepts of enriched category theory.}
  Reprint of the 1982 original [Cambridge Univ. Press, Cambridge]
  Repr. Theory Appl. Categ. No. 10 (2005), vi+137 pp.

  \bibitem{Leinster2}
  Tom Leinster.
  \textit{The magnitude of metric spaces.}
  Doc. Math. 18 (2013), 857--905.

  \bibitem{Leinster1}
  Tom Leinster.
  \textit{The magnitude of a graph.}
  Math. Proc. Cambridge Philos. Soc. 166 (2019), no. 2, 247--264.

  \bibitem{Petersen}
  The Automorphism Group of the Petersen Graph is Isomorphic to $S_5$.
  https://arxiv.org/abs/2012.02942

\end{thebibliography}
\end{frame}

\appendix

\begin{frame}{Motivation}
  集合の要素数や, ベクトル空間の次元, 位相空間のオイラー標数などの不変量は, magnitude という一つの定義で統一的に扱うことができる.

  Magnitude は, Enriched categoryの上定義されるもの.

  距離空間は, $[0, \infty]$-enriched category として考えることができる.

  グラフを距離空間として見ることでグラフにも一般的な不変量を構成できる.
\end{frame}


\begin{frame}{Proposition 1 の証明の概略}
  \begin{proof}
    $\tilde{w}_G:V(G) \rightarrow \ZZ[q]$ を, 定理1行目で $x_0$ を固定したものとする↓
    \[
      \tilde{w}_G(x) = \sum_{k=0}^{\infty} (-1)^k \sum_{x = x_0 \neq x_1 \neq \cdots \neq x_k} q^{d(x_0, x_1) + d(x_1, x_2) + \dots + d(x_{k-1}, x_k)}.
    \]
    これは weighting equation を満たす.
    $x_0$ の固定を解除して足すことで題意を得る.
  \end{proof}
\end{frame}


\begin{frame}{Petersen グラフの Autom Grp が 5-dim synmetric group に同型であること(ここは要修正)}
  \begin{columns}[c]
    \begin{column}{0.5\textwidth}
      \includegraphics[height=0.9\textheight, keepaspectratio]{Referrence/Petersen_proof.pdf}
    \end{column}
    \begin{column}{0.5\textwidth}
      右図で, 黄色, 赤色, オレンジ色, 青色, 緑色の頂点をセットにして, 任意に5つの組を入れ替えることが可能.
      よって, $S_5 \subset \aut (G)$ (部分群)
    \end{column}
    
  \end{columns}

\end{frame}

\begin{frame}{他の係数が持っている情報}
  一般的に, $c_0 \geq 0, c_1 \leq 0, c_2 \geq 0$ は成り立つ. $c_2 \geq 0$ は
  \[
    c_2 = |\{ (x,y,z) \mid d(x,y) = d(y,z) = 1 \}| - |\{ (x,y) \mid d(x,y) = 2 \}|
  \]
  から成り立つ. しかし, $c_3 \leq 0$ は成り立たない. これは Petersen グラフ $G$ の例を見ればわかる. 他のグラフについても以下に併せて掲載.
  \begin{center}
    \includegraphics[width=\linewidth]{Tables/out/magnitude_coefficient.pdf}  
  \end{center}
\end{frame}

\begin{frame}{サイクルグラフと完全2部グラフの Magnitude}
  \begin{align*}
    \# C_{2m} &= \frac{n(1-q)}{(1+q)(1-q^m)} \\
    \# C_{2m-1} &= \frac{n(1-q)}{1+q-2q^m} \\
    \# K_{m,n} &= \frac{(m+n) - (2mn - m - n)q}{(1+q)(1-(m-1)(n-1)q^2)}
  \end{align*}
\end{frame}

\begin{frame}{条件なしでは, 主結果は得られない}
  \begin{lemma}
    グラフ不変量 $\Phi$ が, cartesian product に関して積に分解する,かつ, 主定理を無条件に満たすとき, $\Phi (G) = |V(G)|$ となる.
  \end{lemma}
  Magnitude は cartesian product に関して積に分解するので, 主定理を満たすには条件が必要. 例えば, $C_3$ を下図のように分けた場合, 主定理は成り立たない.
  \begin{center}
    \begin{tikzpicture}[
        dot/.style={circle, fill=black, inner sep=2pt, outer sep=0pt},
        thick
      ]
      \begin{scope}[shift={(0, 0)}]
        \node[dot] (a1) at (0, 0.866) {};
        \node[dot] (a2) at (-0.5, 0) {};
        \node[dot] (a3) at (0.5, 0) {};
        \draw (a1) -- (a2);
        \draw (a1) -- (a3);
        \draw[red, very thick] (a2) -- (a3);
      \end{scope}

      \node at (1.25, 0.433) {\Large $\rightarrow$};

      \begin{scope}[shift={(2.5, 0)}]
        \node[dot] (b1) at (0, 0.866) {};
        \node[dot] (b2) at (-0.5, 0) {};
        \node[dot] (b3) at (0.5, 0) {};
        \draw (b1) -- (b2);
        \draw[red, very thick] (b2) -- (b3);
      \end{scope}

      \node at (3.75, 0.433) {\Large $\cup$};

      \begin{scope}[shift={(5, 0)}]
        \node[dot] (c1) at (0, 0.866) {};
        \node[dot] (c2) at (-0.5, 0) {};
        \node[dot] (c3) at (0.5, 0) {};
        \draw (c1) -- (c3);
        \draw[red, very thick] (c2) -- (c3);
      \end{scope}
    \end{tikzpicture}
  \end{center}
  
  \begin{align*}
    \# C_3 = \frac{3}{1 + 2q} \neq \frac{4 - 2q}{1 + q} = 2 \cdot \# T - \# C_2
  \end{align*}
  

\end{frame}


\begin{frame}{端折った計算のアイデアを記載する}

\end{frame}

\begin{frame}{torsionについて}
  今回扱ったグラフの magnitude homology には, torsion は存在しない. 

  しかし, torsion を持つグラフも存在する.
  \begin{center}
    \includegraphics[width=\linewidth]{Img/torsion.png}
    TORSION IN THE MAGNITUDE HOMOLOGY OF GRAPHS
    
    RADMILA SAZDANOVICAND VICTOR SUMMERS
    
    https://arxiv.org/pdf/1912.13483
    
    より (2019)
  \end{center}
\end{frame}

\begin{frame}{4頂点以下のグラフについて / Magnitude は connected components の個数 は特定できない}
  4頂点以下の全てのグラフについて調べた結果, Magnitudeが同じもののペアは以下の通りであった:
\end{frame}

\begin{frame}{categorificationについて}
  categorificationなので, 実際はfunctorの構造が入る.
  グラフ写像 $f : V(G) \rightarrow V(H)$ があると, magnitude complex の間にも写像が誘導され,それが chain map なので, homology 間にも写像が誘導される. それを $f_{\#}$と書くと,$\#$ は\textrm{Graphs} から \textrm{BAB} への functor となる.
\end{frame}

\begin{frame}{実際に計算する}
  \begin{lemma}[Leinster, 2014]\label{vertex_transitive_magnitude}
    グラフ $G$ が vertex-transitive ならば, 任意の頂点 $x \in V(G)$ に対し,
    \[
    \# G(q) = \frac{|V(G)|}{\sum_{y \in V(G)} q^{d(x,y)}}.
    \]
  \end{lemma}
  ※ \textbf{vertex-transitive}: 作用 $\aut (G) \curvearrowright V(G)$ の軌道が唯1つであるグラフ.
  \begin{proof}
    $S : V(G) \rightarrow \QQ(q) ; x \mapsto \sum_{y \in V(G)} q^{d(x,y)}$ とおくと, 仮定より, $S(x)$ の値は $x$ によらず, これを $S$ とすると, 定値関数 $\frac{1}{S}$ は weighting equation を満たす. よって, 
    \[
      \# G(q) = \sum_{x \in V(G)} \frac{1}{S} = \frac{|V(G)|}{S}.
    \]
  \end{proof}
\end{frame}

\end{document}