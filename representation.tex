\documentclass[dvipdfmx, 11pt, notheorems]{beamer}

\usepackage{bxdpx-beamer}
\usepackage{pxjahyper}
\usepackage{amsmath, amssymb, amsthm}
\usepackage{bm}
\usepackage{stmaryrd}
\usepackage{aliascnt}
\usepackage{tikz}
\usepackage{chngcntr}
\usepackage[capitalize]{cleveref}
\usepackage{appendixnumberbeamer}

\renewcommand{\kanjifamilydefault}{\gtdefault}

\newcommand{\tsum}{\mathrm{sum}}
\newcommand{\tadj}{\mathrm{adj}}
\newcommand{\aut}{\mathrm{Aut}}
\newcommand{\isom}{\xrightarrow{\sim}}
\newcommand{\mor}{\mathrm{Mor}}
\newcommand{\rank}{\mathrm{rank}}
\newcommand{\sign}{\mathrm{sign}}
\newcommand{\tor}{\mathrm{Tor}}
\newcommand{\Ker}{\mathrm{Ker}}
\newcommand{\Hom}{\mathrm{Hom}}
\newcommand{\im}{\mathrm{Im}}
\newcommand{\Span}{\mathrm{span}}
\newcommand{\ZZ}{\mathbb{Z}}
\newcommand{\QQ}{\mathbb{Q}}
\newcommand{\NN}{\mathbb{N}}
\newcommand{\RR}{\mathbb{R}}
\newcommand{\CC}{\mathcal{C}}
\newcommand{\VV}{\mathcal{V}}
\newcommand{\calA}{\mathcal{A}}
\newcommand{\calB}{\mathcal{B}}
\newcommand{\ob}{\mathcal{O}\mathrm{b}}
\newcommand{\graphimg}[2][]{
  \includegraphics[#1]{Tikz/Graphs/out/#2}
}

\newcommand{\commutimg}[2][]{
  \includegraphics[#1]{Tikz/Commutative/out/#2}
}
\newcommand{\figcommu}[2][]{
  \begin{center}
    \commutimg[#1]{#2}
  \end{center}
}

\newcommand{\boxprod}{\mathbin{\square}}




\usetheme{AnnArbor}
\usecolortheme{crane}
\usefonttheme{professionalfonts}

\setbeamertemplate{navigation symbols}{}
\setbeamertemplate{theorems}[numbered]

\theoremstyle{definition}
\newtheorem{definition}{Definition}
\newtheorem{theorem}{Theorem}
\newtheorem{lemma}{Lemma}
\newtheorem{example}{Example}
\theoremstyle{plain}
\newtheorem{proposition}{Proposition}
\newtheorem{corollary}{Corollary}

\title[Magnitude Homology]{グラフのMagnitude\\そのCategorificationとしてのMagnitudeホモロジー}
\subtitle{2025年度 卒業研究発表}
\author[Kensho Yachi]{谷内 賢翔(やち けんしょう)}
\institute[YNU]{横浜国立大学 理工学部 数理科学EP}
\date{2025年2月17-19日}

\begin{document}

\setlength\abovedisplayskip{3pt}
\setlength\belowdisplayskip{3pt}
\setlength\abovedisplayshortskip{-8pt}
\setlength\belowdisplayshortskip{2pt}

\begin{frame}
  \titlepage
\end{frame}

\begin{frame}{目次}
  \tableofcontents
\end{frame}

\section{前提知識 1}

\begin{frame}{グラフのMagnitude}
  $G=(V(G),E(G))$ を, ループと多重辺を許さない有限グラフとする.
  $G$ の similarity matrix $Z_G (q)$を, その成分が $Z_G(x,y) = q^{d(x,y)}$ である $|V(G)| \times |V(G)|$-行列として定義する. ここで,$q$ は変数であり,$q^{\infty} = 0$ とする.
  このとき,$Z_G(q)$ の対角成分は全て $1$ なので,$detZ_G$ の定数項は$1$であり, よって  $Z_G$ は可逆である.
  \begin{definition}[magnitude]\label{magnitude}
    グラフ $G$ の magnitude は次のように定義される:
    \[
      \# G(q) = \sum_{x,y \in V(G)} (Z_G(q)^{-1})(x,y) \in \mathbb{Q}(q).
    \]
    (逆行列の全成分の和)
  \end{definition}
  
\end{frame}

\begin{frame}{例: 完全グラフ $K_3$}
  \begin{columns}
    \begin{column}{0.4\textwidth}
      \centering
      \begin{tikzpicture}
        \foreach \i in {1,2,3}
          \coordinate (v\i) at (90+120*\i:1.2);
        \draw (v1) -- (v2) -- (v3) -- cycle;
        \foreach \i in {1,2,3}
          \fill (v\i) circle (2pt);
      \end{tikzpicture}
      \\ 完全グラフ $K_3$
    \end{column}
    \begin{column}{0.6\textwidth}
      $K_3$ のsimilarity matrixは次のようになる:
      \[
        Z_{K_3} = \begin{pmatrix} 1 & q & q \\ q & 1 & q \\ q & q & 1 \end{pmatrix}
      \]
      逆行列 $Z_{K_3}^{-1}$ の成分和を計算すると:
      \[
        \# K_3(q) = \frac{3}{1+2q}.
      \]
    \end{column}
  \end{columns}

  \vspace{0.5em}
  一般に, \[\# K_n (q) = \frac{n}{1 + (n-1)q} \quad \text{(後ほど証明)}.\]
\end{frame}

\begin{frame}{Magnitudeを $\ZZ \llbracket q \rrbracket$ の元として見る}
  定義より, 
  \[
    \# G(q) = \tsum(Z_G(q)^{-1}) = \frac{\tsum(\tadj(Z_G(q)))}{\det(Z_G(q))}
  \]
  である.ここで,$\tsum$ は行列の全成分の和, $\tadj$ は余因子行列を表す.
  これと, $\det(Z_G)$ の定数項が $1$ であることより, $\# G(q)$ は $\ZZ \llbracket q \rrbracket$ に属する. 

  $\ZZ \llbracket q \rrbracket$ で見ることが今後重要となる.

  \vspace{0.5em}
  ※ $\ZZ \llbracket q \rrbracket = \{ \sum_{n=0}^\infty a_n q^n \mid a_n \in \ZZ \}$ (整係数形式的冪級数環)
\end{frame}

\begin{frame}{Weighting}
  \begin{definition}[weighting]
    グラフ $G$ の各頂点 $x$ に対し, 
    \[
      w_G(x)(q) = \sum_{y \in V(G)}(Z_G(q))^{-1}(x,y)
    \]
    で関数 $w_G: V(G) \to \QQ (q)$ を定義し, これを $G$ の weighting と呼ぶ.
  \end{definition}
  定義より, $\# G (q) = \sum_{x \in V(G)} w_G(x)(q)$ である.

  また, $G$ の各頂点 $x$ に対し, weighting $w_G(x)$ は次の方程式を満たす:
   \[
    \sum_{y \in V(G)} q^{d(x,y)} w_G(y) = 1 \quad \text{ for } x \in V(G).
  \]
  これを, weighting equation と呼ぶ. $Z_G(q)$ の可逆性より, weighting equation を満たす他の関数 $V(G) \to \QQ(q)$ は存在しない.
\end{frame}

\begin{frame}{実際の計算例}
  \begin{lemma}
    グラフ $G$ が vertex-transitive ならば, 任意の頂点 $x \in V(G)$ に対し,
    \[
    \# G(q) = \frac{|V(G)|}{\sum_{y \in V(G)} q^{d(x,y)}}.
    \]
  \end{lemma}
  \vspace{0.5em}
  ※ vertex-transitive: 作用 $\aut (G) \curvearrowright V(G)$ の軌道が唯1つであるグラフ.
  \begin{proof}
    $S : V(G) \rightarrow \QQ(q) ; x \mapsto \sum_{y \in V(G)} q^{d(x,y)}$ とおくと, 仮定より, $S(x)$ の値は $x$ によらず, これを $S$ とすると, 定値関数 $\frac{1}{S}$ は weighting equation を満たす. よって, 
    \[
      \# G(q) = \sum_{x \in V(G)} \frac{1}{S} = \frac{|V(G)|}{S}.
    \]
  \end{proof}
\end{frame}

\begin{frame}{実際の計算例: 完全グラフ $K_n$}
  $n$頂点完全グラフ $K_n$ は vertex-transitive である.
  Lemma 1 において, $S = 1 + (n-1)q$ であるから,
  \[ 
    \# K_n(q) = \frac{n}{1 + (n-1)q}.
  \]
\end{frame}

\begin{frame}{実際の計算例: Petersenグラフ}
  \begin{columns}[c]
    \begin{column}{0.45\textwidth}
      \centering
      \graphimg[width=\linewidth]{petersen.pdf}
    \end{column}
    \begin{column}{0.55\textwidth}
      $G$ を左図のグラフ (Petersen グラフ) とすると, $G$ もvertex-transitive となる (回転と内外の入れ替え). $S = 1 + 3q + 6q^2$ なので, 
      \[
        \# G (q) = \frac{10}{1 + 3q + 6q^2}. 
      \]
      \bigskip
    \end{column}
  \end{columns}
\end{frame}

\begin{frame}{組み合わせ的表現}
  \begin{proposition}
    $G$ をグラフとするとき, 
    \[
    \begin{split}
      \# G(q) &= \sum_{k=0}^{\infty} (-1)^k \sum_{x_0 \neq \cdots \neq x_k} q^{d(x_0, x_1) + \dots + d(x_{k-1}, x_k)} \\
      &= \sum_{n=0}^{\infty} c_n q^n
    \end{split}
    \]
    が成り立つ.ここで, 
    \begin{multline*}
      c_n = \sum_{k=0}^{n} (-1)^k \Bigl| \Bigl\{ (x_0, \dots, x_k) \Bigm| x_0 \neq \cdots \neq x_k, \\
       d(x_0, x_1) + \dots + d(x_{k-1}, x_k) = n \Bigr\} \Bigr|.
    \end{multline*}
\end{proposition}
\end{frame}

\begin{frame}{Proposition 1 の証明の概略}
  \begin{proof}
    $\tilde{w}_G:V(G) \rightarrow \ZZ[q]$ を, 定理1行目で $x_0$ を固定したものとする↓
    \[
      \tilde{w}_G(x) = \sum_{k=0}^{\infty} (-1)^k \sum_{x = x_0 \neq x_1 \neq \cdots \neq x_k} q^{d(x_0, x_1) + d(x_1, x_2) + \dots + d(x_{k-1}, x_k)}.
    \]
    これは weighting equation を満たす.
  \end{proof}
\end{frame}

\begin{frame}{Proposition 1 からわかること}
  \begin{corollary}
    $G$ をグラフとするとき, 
    \[
      |V(G)| = \# G(0), |E(G)| = - \frac{1}{2} \left. \frac{d}{dq} \# G(q) \right|_{q=0}.
    \]   
  \end{corollary}
  \begin{proof}
    Proposition 1 を見ると, 以下を確認できる:
    \[
      c_0 = |V(G)|, c_1 = -2|E(G)|.
    \]
  \end{proof}
\end{frame}

\begin{frame}{Convexとprojection}
  \begin{definition}[convex]
    $X$ をグラフ, $U$ を $X$ の部分グラフとする.
    $U$ が $X$ で凸であるとは, 任意の頂点 $x,y \in V(U)$ に対し, $d_U(x,y) = d_X(x,y)$ が成り立つことをいう.
  \end{definition}
  \begin{definition}[projection]
    $X$ をグラフ, $U$ を $X$ で凸な $X$ の部分グラフとする. また, 
    $V_U(X) = \{ v \in V(X) \mid d_X(v,u) < \infty \text{ for some } u \in V(U) \}$ とする($U$ に接続されている頂点全体). 
    
    $X$ \textit{projects to $U$} とは, 
    任意の $x \in V_U(X)$ に対し, ある $u' \in V(U)$ が存在して, 任意の $u \in V(U)$ について $d_X(x,u) = d_X(x, u') + d_X(u',u)$ が成り立つことをいう. 各 $x \in V_U(X)$ に対し, そのような $u'$ を1つ固定し, これを $\pi(x)$ と表す.
  \end{definition}
\end{frame}

\begin{frame}{projectの例}
  (赤に接続された)すべての頂点に対し,赤の中から1点選んでfactor throughすることで, 赤の全ての点との距離が最短になるようにできる.
  
  上記の $u'$ は一意に定まる.
  \begin{center}
    \graphimg[height=4cm, keepaspectratio]{K_mn_2.pdf}
    \hspace{1cm}
    \graphimg[height=4cm, keepaspectratio]{Grid_Graph_Red_U.pdf}
  \end{center}
\end{frame}


\section{主結果 1}
\begin{frame}{主結果}
  \begin{theorem}
    $X$ をグラフ,  $G,H$ を $X$ の部分グラフとし, $X = G \cup H$ とする.
    $G \cap H$ が $X$ で凸であり, $H$ projects to $G \cap H$ ならば,
    \[
      \# X = \# G + \# H - \#(G \cap H).
    \]
  \end{theorem}

  上記の状況を満たす pair $(X;G,H)$ を, \textit{projecting decomposition} と呼ぶ.
  
  \begin{proof}
    $w_G + w_H - w_{G \cap H}$ が$X$ のweighting equationを満たすので, $w_X = w_G + w_H - w_{G \cap H}$ が導かれる.
    weighting equationを満たすことは, 計算により確認できる.
  \end{proof}
\end{frame}

\begin{frame}{計算の様子}
  \tiny
  \begin{align*}
    d(g, u) + d(u, h) &= d(g, u) + d(u, \pi(h)) + d(\pi(h), h) \\
    &\geq d(g, \pi(h)) + d(\pi(h), h) \\
    &\geq d(g, h) \\
    &= d(g, u) + d(u, h)
  \end{align*}
  means
  \[
    d(g, h) = d(g, \pi(h)) + d(\pi(h), h).
  \]
  If $x \in V(G)$, 
  \begin{align*}
    &\sum_{g \in V(G)}q^{d(g, x)} w_G(g) + \sum_{h \in V(H)} q^{d(h, x)} w_H(h) - \sum_{u \in V(G \cap H)} q^{d(u, x)} w_{G \cap H}(u) \\
    &= 1 + \sum_{h \in V(H)} q^{d(h, x)} w_H(h) - \sum_{u \in V(G \cap H)} q^{d(u, x)} \sum_{h \in \pi^{-1}(u)} q^{d(h, u)}w_H(h) \\
    &= 1 + \sum_{h \in V_{G \cap H}(H)} q^{d(h, x)} w_H(h) - \sum_{h \in V_{G \cap H}(H)} q^{d(x, \pi(h)) + d(\pi(h), h)} w_H(h) \\
    &= 1 + \sum_{h \in V_{G \cap H}(H)} q^{d(h, x)} w_H(h) - \sum_{h \in V_{G \cap H}(H)} q^{d(h, x)} w_H(h) \\
    &= 1.
  \end{align*}
  If $x \in V_{G\cap H}(H)$, 
  \begin{align*}
    &\sum_{g \in V(G)}q^{d(g, x)} w_G(g) + \sum_{h \in V(H)} q^{d(h, x)} w_H(h) - \sum_{u \in V(G \cap H)} q^{d(u, x)} w_{G \cap H}(u) \\
    &= \sum_{g \in V(G)}q^{d(g, \pi(x)) + d(\pi(x), x)} w_G(g) + 1 - \sum_{u \in V(G \cap H)} q^{d(u, \pi(x)) + d(\pi(x), x)} w_{G \cap H}(u)
  \end{align*}
\end{frame}

\begin{frame}{主定理の使用例 : Bull Graph}
  \begin{columns}[c]
    \begin{column}{0.5\textwidth}
      \centering
      \begin{tikzpicture}[
        scale=1.5,
        dot/.style={circle, fill=black, inner sep=2pt, outer sep=1pt}, 
        thick
      ]
        \node[dot] (v1) at (0, 0) {};
        \node[dot] (v2) at (1, 0) {};
        \node[dot] (v3) at (0.5, -0.866) {};
        \node[dot] (v4) at (-0.5, 0.5) {};
        \node[dot] (v5) at (1.5, 0.5) {};
        \draw (v1) -- (v2) -- (v3) -- (v1);
        \draw (v1) -- (v4);
        \draw (v2) -- (v5);
      \end{tikzpicture}
    \end{column}

    \begin{column}{0.5\textwidth}      
      \bigskip
      左図のグラフ $G$ 場合, $G = K_2 \vee C_3 \vee K_2$ とみなせる ($\vee$ は, 1点の同一視).
      この場合, 1点和は projecting decomposition を与えるので, 主定理を2回適応することにより,
      \[
        \# G = \# C_3 + 2 \cdot \# K_2 - 2 \cdot \# K_1.
      \]
      となる. 
    \end{column}
    
  \end{columns}
\end{frame}

\section{前提知識 2}

\begin{frame}{Magnitude complex}
  \begin{definition}
    グラフ $G$, 非負整数 $k,l$ に対し, 自由アーベル群
    \[
    \begin{split}
      MC_{k,l}(G) = \Span_{\ZZ} \bigl\{ (x_0, \dots, x_k) \bigm| x_i \in V(G), x_0 \neq \cdots \neq x_k, \\
      l = d(x_0, x_1) + \cdots + d(x_{k-1}, x_k) \bigr\}
    \end{split}
    \]
    を定め, boundary $\partial : MC_{k,l}(G) \rightarrow MC_{k-1,l}(G)$ を
    \[
      \begin{split}
        &\partial = \sum_{i=1}^{k-1} (-1)^{i-1} \partial_i,~\partial_i (x_0, \dots, x_k) = \\
        &\begin{cases}
          (x_0, \dots, \hat{x_i}, \dots, x_k) & \text{if } l(x_0, \dots, \hat{x_i}, \dots, x_k) = l(x_0, \dots, x_k), \\
          0 & \text{otherwise}.
        \end{cases}
      \end{split}
      \]
    で定義する.
  \end{definition}

  \vspace{0.6em}
  このように定義された $\partial$ は, 2回適用すると0になり, これにより各 $l$ に対し, 鎖複体 $(MC_{*,l}(G), \partial)$ が得られる.
  
\end{frame}

\begin{frame}{Magnitude homology}
  \begin{definition}
    グラフ $G$ の \textit{magnitude homology} とは, homology
    \[
      MH_{*, l}(G) = H_{\partial}(MC_{*, l}(G))
    \]
    のことである. 各 $l$ について得られる.
  \end{definition}
\end{frame}



\begin{frame}{Magniude との関係}
  \begin{theorem}
    $G$ をグラフとするとき, 
    \[
      \sum_{k,l} (-1)^k \rank (MH_{k,l}(G)) q^l = \# G(q).
    \]
  \end{theorem}
  \begin{proof}
    各 $k, l$ について,
    \[
      \rank (MC_{k,l}(G))q^l = \sum_{x_0 \neq \cdots \neq x_k} q^{d(x_0, x_1) + \cdots + d(x_{k-1}, x_k)}
    \]
    であることや, 複体のオイラー標数に関する定理, Proposition 1 を利用する.
  \end{proof}
\end{frame}

\begin{frame}{Magnitude Homology の例}
  
\end{frame}


\begin{frame}{Enriched category の magnitude}
  $(\VV, \otimes, I, \alpha, \lambda, \rho, |\cdot|)$ をモノイダル圏, $(\calA, m, j)$ を 有限のオブジェクトを持つ $\VV$-enriched category とする.

  \begin{definition}{magnitude}

    
  \end{definition}
\end{frame}

\section{主定理 2}
\begin{frame}{Enriched category から見た グラフ の magnitude}
  \begin{theorem}
    $G$ をグラフとするとき, $G$ には一般化距離空間の構造が入る. その上で $G$ を $[0, \infty]$-enriched category として考えると, その magnitude は, 先ほど定義したグラフの magnitude と一致する.
  \end{theorem}

  \begin{proof}
    グラフの頂点集合 $V(G)$ に,最短経路が定める距離 $d_G$ を入れることで,$V(G)$ は一般化距離空間となる. これを $G$ の距離とする.

    モノイダル圏 $([0, \infty], +, 0, \leq, id, |\cdot|)$ には, 部分的に定義されるモノイド準同型 $| \cdot | : Z \to \QQ(q)$ を $|x| = q^x$ が付加されており, $G$ の距離が $Z$ に値をとることから $G$ は $[0, \infty]$-enriched categoryの構造を持つ. (ただし, $Z = \ZZ_{\geq 0} \cup \{ \infty \}$). これにより \cref{magnitude} と一致させることができる.
  \end{proof}
  
\end{frame}



\section{今後の課題}

\begin{frame}{まとめと今後の課題}
  \textbf{まとめ:}
  \begin{itemize}
    \item Magnitude の復習と Magnitude Homology の定義.
    \item オイラー標数との関係(Categorification)の確認.
    \item Projecting Decomposition における Mayer-Vietoris 完全列の導入.
  \end{itemize}

  \textbf{今後の課題:}
  \begin{itemize}
    \item \textbf{Whitney Twist:} ツイスト操作は Magnitude Homology を保存するか?
    \item \textbf{Diagonal Graphs:} $k \neq l$ で $MH_{k,l}=0$ となるグラフの特徴付け(例: 正二十面体グラフ).
  \end{itemize}
\end{frame}

\begin{frame}{参考文献}
  \begin{thebibliography}{9}

  \bibitem{HepworthWillerton}
  Richard Hepworth and Simon Willerton.
  \textit{Categorifying the magnitude of a graph.}
  Homology Homotopy Appl. 19 (2017), no. 2, 31--60.

  \bibitem{Kelly}
  G. M. Kelly.
  \textit{Basic concepts of enriched category theory.}
  Reprint of the 1982 original [Cambridge Univ. Press, Cambridge]
  Repr. Theory Appl. Categ. No. 10 (2005), vi+137 pp.

  \bibitem{Leinster2}
  Tom Leinster.
  \textit{The magnitude of metric spaces.}
  Doc. Math. 18 (2013), 857--905.

  \bibitem{Leinster1}
  Tom Leinster.
  \textit{The magnitude of a graph.}
  Math. Proc. Cambridge Philos. Soc. 166 (2019), no. 2, 247--264.

  \bibitem{Petersen}
  The Automorphism Group of the Petersen Graph is Isomorphic to $S_5$.
  https://arxiv.org/abs/2012.02942

\end{thebibliography}
\end{frame}

\appendix

\begin{frame}{Motivationの話を書く}
  
\end{frame}

\begin{frame}{monoidal category}
  \begin{definition}
    圏 $\VV$ とそのobject $I$, 関手 $\otimes : \VV \times \VV \to \VV$, 自然変換 $\alpha : \otimes \circ (\otimes \times id_{\VV} \Rightarrow \otimes \circ (id_{\VV} \times \otimes)), \lambda : I \otimes - \Rightarrow id_{\VV}, \rho : - \otimes I \Rightarrow id_{\VV}$ に対し, pair $(\VV, \otimes, I, \alpha, \lambda, \rho)$ が \textit{monoidal category} であるとは, 以下の性質を満たすものをいう:
  \end{definition}
\end{frame}

\begin{frame}{Enriched Category}
  \begin{definition}
    $(\VV, I, \otimes, \alpha, \lambda, \rho)$ を monoidal categoryとする. 圏 $\calA$, a collection of morphisms $m, j$ に対し, pair $(\calA, m, j)$ が \textit{$\VV$-enriched category} であるとは, 以下の性質を満たすものをいう:
  \end{definition}
\end{frame}

\begin{frame}{Enriched Category の例}

\end{frame}

\begin{frame}{Petersen グラフの Autom Grp が 5-dim synmetric group に同型であること(ここは要修正)}
  \begin{columns}[c]
    \begin{column}{0.5\textwidth}
      \includegraphics[height=0.9\textheight, keepaspectratio]{Referrence/Petersen_proof.pdf}
    \end{column}
    \begin{column}{0.5\textwidth}
      右図で, 黄色, 赤色, オレンジ色, 青色, 緑色の頂点をセットにして, 任意に5つの組を入れ替えることが可能.
      よって, $S_5 \subset \aut (G)$ (部分群)
    \end{column}
    
  \end{columns}

\end{frame}

\begin{frame}{Magnitude では, connected components の個数 は特定できない}
  Remark 2.2.12, 2.2.13 あたりをまとめましょう. 
  簡単で非自明なグラフいくつかについて、3次くらいまでの一覧表を用意する.
\end{frame}

\begin{frame}{条件なしでは, 主結果は得られない}
  Lemma 2.3.3 や cartesian product 周りの話をまとめる.
\end{frame}


\begin{frame}{端折った計算のアイデアを記載する}

\end{frame}

\end{document}