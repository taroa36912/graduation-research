\documentclass[dvipdfmx, 11pt, notheorems]{beamer}

\usepackage{bxdpx-beamer}
\usepackage{pxjahyper}
\usepackage{amsmath, amssymb, amsthm}
\usepackage{bm}
\usepackage{stmaryrd}
\usepackage{aliascnt}
\usepackage{tikz}
\usepackage{chngcntr}
\usepackage[capitalize]{cleveref}

\renewcommand{\kanjifamilydefault}{\gtdefault}

\newcommand{\tsum}{\mathrm{sum}}
\newcommand{\tadj}{\mathrm{adj}}
\newcommand{\aut}{\mathrm{Aut}}
\newcommand{\isom}{\xrightarrow{\sim}}
\newcommand{\mor}{\mathrm{Mor}}
\newcommand{\rank}{\mathrm{rank}}
\newcommand{\sign}{\mathrm{sign}}
\newcommand{\tor}{\mathrm{Tor}}
\newcommand{\Ker}{\mathrm{Ker}}
\newcommand{\Hom}{\mathrm{Hom}}
\newcommand{\im}{\mathrm{Im}}
\newcommand{\ZZ}{\mathbb{Z}}
\newcommand{\QQ}{\mathbb{Q}}
\newcommand{\NN}{\mathbb{N}}
\newcommand{\RR}{\mathbb{R}}
\newcommand{\CC}{\mathcal{C}}
\newcommand{\VV}{\mathcal{V}}
\newcommand{\calA}{\mathcal{A}}
\newcommand{\calB}{\mathcal{B}}
\newcommand{\ob}{\mathcal{O}\mathrm{b}}


\usetheme{AnnArbor}
\usecolortheme{crane}
\usefonttheme{professionalfonts}

\setbeamertemplate{navigation symbols}{}
\setbeamertemplate{theorems}[numbered]

\theoremstyle{definition}
\newtheorem{definition}{定義}
\newtheorem{theorem}{定理}
\newtheorem{lemma}{補題}
\newtheorem{example}{例}
\theoremstyle{plain}
\newtheorem{proposition}{命題}
\newtheorem{corollary}{系}

\title[Magnitude Homology]{グラフのMagnitude\\そのCategorificationとしてのMagnitudeホモロジー}
\subtitle{2025年度 卒業研究発表}
\author[Kensho Yachi]{谷内 賢翔(やち けんしょう)}
\institute[YNU]{横浜国立大学 理工学部 数理科学EP}
\date{2025年2月17-19日}

\begin{document}

\begin{frame}
  \titlepage
\end{frame}

\begin{frame}{目次}
  \tableofcontents
\end{frame}

\section{前提知識 (Introduction)}

\begin{frame}{グラフのMagnitude}
  $G=(V,E)$ を, loopとmultiple edgeを許さない有限グラフとする.
  $G$ の similarity matrix $Z_G$を, その成分が $(Z_G)_{xy} = q^{d(x,y)}$ である $|V| \times |V|$-行列として定義する.
  このとき,$detZ_G$ の定数項は$1$であるため, $Z_G$は可逆である.
  \begin{definition}[Magnitude]
    グラフ $G$ の magnitude は次のように定義される:
    \[
      \# G(q) = \sum_{x,y \in V} (Z_G^{-1})_{xy} \in \mathbb{Q}(q).
    \]
    (逆行列の全成分の和)
  \end{definition}
  
  \vspace{0.5em}
  $q^{\infty} = 0$ とする.
\end{frame}

\begin{frame}{例: 完全グラフ $K_3$}
  \begin{columns}
    \begin{column}{0.4\textwidth}
      \centering
      \begin{tikzpicture}
        \foreach \i in {1,2,3}
          \coordinate (v\i) at (90+120*\i:1.2);
        \draw (v1) -- (v2) -- (v3) -- cycle;
        \foreach \i in {1,2,3}
          \fill (v\i) circle (2pt);
      \end{tikzpicture}
      \\ 完全グラフ $K_3$
    \end{column}
    \begin{column}{0.6\textwidth}
      類似度行列は次のようになる:
      \[
        Z_{K_3} = \begin{pmatrix} 1 & q & q \\ q & 1 & q \\ q & q & 1 \end{pmatrix}
      \]
      逆行列 $Z_{K_3}^{-1}$ の成分和を計算すると:
      \[
        \# K_3(q) = \frac{3}{1+2q}.
      \]
    \end{column}
  \end{columns}

  \vspace{0.5em}
  一般に, \[\# K_n = \frac{n}{1 + (n-1)q}\].
\end{frame}

\begin{frame}{Magnitudeを $\ZZ \llbracket q \rrbracket$ の元として見る}
  定義より, 
  \[
    \# G(q) = \tsum(Z_G(q)^{-1}) = \frac{\tsum(\tadj(Z_G(q)))}{\det(Z_G(q))}
  \]
  である.ここで,$\tsum$ は行列の全成分の和, $\tadj$ はadjoint行列を表す.
  これと, $\det(Z_G)$ の定数項が $1$ であることより, $\# G(q)$ は $\ZZ \llbracket q \rrbracket$ に属する. 

  $\ZZ \llbracket q \rrbracket$ で見ることが今後重要となる.

  \vspace{0.5em}
  ※ $\ZZ \llbracket q \rrbracket = \{ \sum_{n=0}^\infty a_n q^n \mid a_n \in \ZZ \}$ (整係数形式的冪級数環)
\end{frame}

\begin{frame}{Magnitudeとweighting}
  \begin{definition}[weighting]
    グラフ $G$ の各頂点 $x$ に対し, 次の方程式を満たす数値 $w_G(x) \in \QQ(q)$ の組 $(w_G(x))_{x \in V(G)}$ を \textbf{weighting} と呼ぶ:
    \[
      \sum_{y \in V(G)} q^{d(x,y)} w_G(y) = 1 \quad (\forall x \in V(G)).
    \]
  \end{definition}
\end{frame}

\begin{frame}{定義: Magnitude鎖複体}
  長さ $l$ で次数付けられた鎖複体 (chain complex) $MC_{*, l}(G)$ を定義する.

  \begin{definition}
    生成元: $x_i \neq x_{i+1}$ を満たす頂点の列 $(x_0, \dots, x_k)$.
    \\
    微分 $\partial = \sum (-1)^i \partial_i$:
    \[
      \partial_i(x_0, \dots, x_k) = 
      \begin{cases}
        (x_0, \dots, \widehat{x_i}, \dots, x_k) & \text{もし長さが保存されるなら} \\
        0 & \text{それ以外}
      \end{cases}
    \]
    (頂点 $x_i$ を抜いても最短経路性が保たれる場合のみ残る)
  \end{definition}
  
  \alert{Magnitude Homology $MH_{k,l}(G)$} は、この複体のホモロジーである.
\end{frame}

\begin{frame}{Categorification (圏化)}
  Magnitude Homology は Magnitude を「圏化」する.

  \begin{theorem}[Hepworth-Willerton]
    $MH(G)$ の次数付きオイラー標数は、元の magnitude を復元する:
    \[
      \sum_{k,l} (-1)^k \operatorname{rank}(MH_{k,l}(G)) q^l = \# G(q).
    \]
  \end{theorem}

  これは、$\#G$ の持つ性質が、$MH(G)$ のより深い構造的性質の「影」であることを意味する.
\end{frame}

\section{主結果}

\begin{frame}{Mayer-Vietoris 完全列}
  グラフの分解 $G = A \cup B$ から、$MH(G)$ を計算したい.

  \begin{theorem}[Mayer-Vietoris 完全列]
    もし分解 $(G; A, B)$ が \textbf{Projecting Decomposition (射影分解)} であるならば、以下の長完全列が存在する:
    \begin{align*}
      \cdots \to MH_n(A \cap B) \to MH_n(A) \oplus MH_n(B) \\
      \to MH_n(G) \to MH_{n-1}(A \cap B) \to \cdots
    \end{align*}
  \end{theorem}
  
  これは包除原理 $\#G = \#A + \#B - \#(A \cap B)$ の圏化版である.
\end{frame}

\begin{frame}{証明のアイデア: Projecting Decomposition}
  グラフにおいて、通常の Mayer-Vietoris 列は必ずしも成立しない.
  我々は \textbf{Projecting (射影)} 条件を課す:
  
  \begin{itemize}
    \item 共通部分 $A \cap B$ は $G$ において \textit{convex (凸)} でなければならない.
    \item $G$ の各頂点は、$A \cap B$ への距離を保存する一意的な「射影」を持たなければならない(最短パスが射影点を通る).
  \end{itemize}

  この条件の下で、鎖複体が適切に分裂し、完全列が導かれる.
\end{frame}


\begin{frame}{動機: 豊穣圏}
  なぜ "Magnitude" と呼ぶのか?
  \begin{itemize}
    \item グラフは、モノイダル圏 $([0, \infty], +, 0)$ 上の豊穣圏とみなせる.
    \item 対象: 頂点
    \item 射 (Hom object): 最短パス距離
  \end{itemize}
  
  \begin{block}{整合性}
    豊穣圏に対する Magnitude の一般定義が、グラフに対する組合せ的な定義(第2節)と一致することを再確認した.
  \end{block}
\end{frame}

\section{今後の課題}

\begin{frame}{まとめと今後の課題}
  \textbf{まとめ:}
  \begin{itemize}
    \item Magnitude の復習と Magnitude Homology の定義.
    \item オイラー標数との関係(Categorification)の確認.
    \item Projecting Decomposition における Mayer-Vietoris 完全列の導入.
  \end{itemize}

  \textbf{今後の課題:}
  \begin{itemize}
    \item \textbf{Whitney Twist:} ツイスト操作は Magnitude Homology を保存するか?
    \item \textbf{Diagonal Graphs:} $k \neq l$ で $MH_{k,l}=0$ となるグラフの特徴付け(例: 正二十面体グラフ).
    \item \textbf{Cohomology:} Magnitude Cohomology への積構造の導入.
  \end{itemize}
\end{frame}

\end{document}