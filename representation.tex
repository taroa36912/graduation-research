\documentclass[dvipdfmx, 11pt, notheorems]{beamer} 
% notheorems: amsthmの定理環境と競合しないようにするオプション

% ==========================================
% パッケージ設定
% ==========================================
\usepackage{bxdpx-beamer} % dvipdfmxでのナビゲーション修正
\usepackage{pxjahyper}    % 日本語のしおり対応
\usepackage{minijs}       % 日本語フォント設定
\usepackage{amsmath, amssymb, amsthm}
\usepackage{tikz}
\usepackage{graphicx}
\usepackage{bm}

% ==========================================
% デザイン・テーマ設定
% ==========================================
\usetheme{Madrid}         % 定番のテーマ(下部に進捗バーが出る)
\usecolortheme{default}   % 色のテーマ(defaultは青系)
% \usecolortheme{beaver}  % 赤系が良ければコメントアウトを外す

% フォント設定(数式をSerifにするなど)
\usefonttheme{professionalfonts} 

% ナビゲーションバーの記号を消す(スライドがすっきりします)
\setbeamertemplate{navigation symbols}{}

% 定理環境の日本語化とスタイル設定
\setbeamertemplate{theorems}[numbered] % 定理に番号を振る
\theoremstyle{definition}
\newtheorem{definition}{定義}
\newtheorem{theorem}{定理}
\newtheorem{lemma}{補題}
\newtheorem{example}{例}
\theoremstyle{plain}
\newtheorem{proposition}{命題}

% ==========================================
% タイトル情報
% ==========================================
\title[Magnitude Homology]{Magnitude Homology of Graphs and the Magnitude as its Categorification}
% [ ]内はフッターに表示される略称
\subtitle{2025年度 卒業研究発表}
\author[Kensho Yachi]{谷地 健将 (Kensho Yachi)}
\institute[YNU]{横浜国立大学 理工学部 数理科学EP}
\date{2026年2月X日}

% ==========================================
% 本文開始
% ==========================================
\begin{document}

% --- タイトルスライド ---
\begin{frame}
  \titlepage
\end{frame}

% --- 目次スライド ---
\begin{frame}{目次}
  \tableofcontents
\end{frame}

% ==========================================
\section{Introduction}
% ==========================================

\begin{frame}{研究の背景 (Motivation)}
  \begin{itemize}
    \item グラフの不変量として \textbf{Magnitude} が知られている
    \item これを圏論的視点(Enriched Category)から捉え直す動きがある
    \item \alert{Magnitude Homology} は、Magnitudeの「オイラー標数化」として定義される
  \end{itemize}

  \vspace{1em}
  \begin{block}{本研究の目的}
    グラフのMagnitude Homologyの性質を調べ、
    特定の条件下でのMayer-Vietoris完全列の存在を示す。
  \end{block}
\end{frame}

% ==========================================
\section{Preliminaries}
% ==========================================

\begin{frame}{準備: Magnitudeの定義}
  グラフ $G=(V,E)$ に対し、そのMagnitudeは以下のように定義される。

  \begin{definition}[Magnitude of Graphs]
    $G$ のsimilarity matrix $Z$ が可逆であるとき、
    \[
      |G| = \sum_{i,j \in V} (Z^{-1})_{ij}
    \]
    と定義する。これは $\mathbb{Q}(q)$ の元である。
  \end{definition}

  \pause % ここでクリック待ちが入ります

  \begin{example}[完全グラフ $K_3$]
    TikZで図を描画したり、画像を貼ったりできます。
    \begin{center}
      \begin{tikzpicture}[scale=0.8]
        \node (a) at (90:1) {$\bullet$};
        \node (b) at (210:1) {$\bullet$};
        \node (c) at (330:1) {$\bullet$};
        \draw (a) -- (b) -- (c) -- (a);
      \end{tikzpicture}
    \end{center}
    このとき $|K_3| = \frac{3}{1+2q}$ となる。
  \end{example}
\end{frame}

% ==========================================
\section{Main Results}
% ==========================================

\begin{frame}{主結果: Magnitude Chain Complex}
  \begin{definition}[Magnitude Chain Complex]
    長さ $k$ のパス $x=(x_0, \dots, x_k)$ を基底とする自由加群 $MC_k(G)$ を考え、
    境界作用素 $\partial$ を以下で定義する:
    \[
      \partial_k(x) = \sum_{i=1}^{k-1} (-1)^i (x_0, \dots, \widehat{x_i}, \dots, x_k)
    \]
  \end{definition}

  \vspace{0.5em}
  これにより、ホモロジー群 $MH_{k,l}(G)$ が構成される。
\end{frame}

\begin{frame}{主結果: Mayer-Vietoris Sequence}
  本研究における主要な定理は以下の通りである。

  \begin{theorem}[Mayer-Vietoris Sequence]
    グラフ $G$ が特定の分解条件(Projecting Decomposition)を満たすとき、
    以下の長完全列が存在する:
    \begin{align*}
      \cdots \to MH_n(A \cap B) \to MH_n(A) \oplus MH_n(B) \\
      \to MH_n(G) \to MH_{n-1}(A \cap B) \to \cdots
    \end{align*}
  \end{theorem}
\end{frame}

% ==========================================
\section{Conclusion}
% ==========================================

\begin{frame}{まとめと今後の課題}
  \begin{columns}[T] % 2段組みにする設定
    \begin{column}{0.48\textwidth}
      \textbf{まとめ}
      \begin{itemize}
        \item Magnitude Homologyの定義を確認した
        \item MV完全列の存在条件を整理した
        \item 具体的なグラフで計算を行った
      \end{itemize}
    \end{column}
    
    \begin{column}{0.48\textwidth}
      \textbf{今後の課題}
      \begin{itemize}
        \item より一般の距離空間への拡張
        \item Path Homologyとの関連性の調査
        \item 応用例の探索
      \end{itemize}
    \end{column}
  \end{columns}
\end{frame}

\begin{frame}[allowframebreaks]{参考文献}
  \bibliographystyle{plain}
  \begin{thebibliography}{99}
    \bibitem{Leinster2013}
    T. Leinster, \textit{The magnitude of metric spaces}, Doc. Math. 18 (2013), 857--905.
    
    \bibitem{Hepworth2017}
    R. Hepworth and S. Willerton, \textit{Categorifying the magnitude of a graph}, Homology Homotopy Appl. 19 (2017), 31--60.
  \end{thebibliography}
\end{frame}

\begin{frame}
  \centering
  \Huge
  ご清聴ありがとうございました
\end{frame}

\end{document}