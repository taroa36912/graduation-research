\documentclass{jarticle}

\setlength{\topmargin}{-1.4cm}
%\setlength{\paperwidth}{14cm}
\setlength{\oddsidemargin}{-30Q}
\setlength{\evensidemargin}{0cm}

\setlength{\columnsep}{2.5zw}

\usepackage{amsmath}
\usepackage{amssymb}
\usepackage{amsthm}
\usepackage{eucal}


%%%%%%%%%%%%%%%%%%%%%%%%%%%%%%%%
%%%%%% 以下の項目を適宜,修正して
%%%%%% 利用してください.
%%%%%%
%%%%%%%%%%%%%%%%%%%%%%%%%%%%%%%%
\newcommand{\TITLE}%
{卒業研究 題目}%卒論タイトルを書く
\newcommand{\eTITLE}%
{英文タイトル}%%英語タイトル
\newcommand{\STNO}%
{22xxxxxx}%学籍番号
\newcommand{\NAME}%
{横浜 みらい}%氏名
\newcommand{\ADVR}%
{横浜 みなと 教授}%指導教員氏名

%%%% 以下は変更しないでください
\newcommand{\DATE}%
{(2026年2月6日)}%〆切
\newcommand{\NENDO}%
{2025年度}%
%%%%%%%%%%%%%%%%%%%%%%%%%%%%%%%%%%

\setlength{\textwidth}{52zw}
\setlength{\textheight}{25cm}


\makeatletter

\def\ps@lrheadf{\ps@empty
  \def\@evenhead{\normalfont%\footnotesize
      \leftmark{}{}\hfil \rightmark{}{}}%
  \let\@oddhead\@evenhead
  \def\@oddfoot{}
  \let\@evenfoot\@oddfoot
  \let\@mkboth\markboth}
\makeatother


\pagestyle{lrheadf}

\markboth{\NENDO \hskip1zw 
数物・電子情報系学科 数理科学EP 
\hskip1zw
卒業研究 \hskip1zw 概要}{}



\begin{document}
%\fontsize{9pt}{14pt}\selectfont
\fontsize{10pt}{14pt}\selectfont


\hfil
\textbf{\Large \TITLE }

\vskip5Q
\hfil
{\large \eTITLE }


\vskip5Q
\hfil 
{\large \STNO \hskip1zw \NAME
\hskip 2zw 指導教員: \ADVR}

\vskip20Q


\noindent
\textbf{1.研究背景・動機}

\NENDO 数理科学EP卒業研究の概要フォーマットです.
用紙はA4用紙1枚です.
最初の1段組の枠に,
日本語の研究題目,英文の研究題目,学籍番号,
氏名,指導教員の氏名を記入してください.

用紙はA4用紙で1ページです.文章本体は2段組です.
文字の
大きさは10ポイント程度です.

全体の構成は,1.研究背景・動機, 
2.主結果, 3.意義・証明のアイデアや方法, 
4.今後の課題, 参考文献, を参考にしてください. 
項目ごとに,
(専門外の人にも概要がわかるように)
わかりやすく,明確に記述してください.
○○○○○○○○○○○○○○○○○○○○○○○○○○○○○○○
○○○○○○○○○○○○○○○○○○○○○○○○○○○
○○○○○○○○○○○○○○○○○○○○○○○○○○○
○○○○○○○○○○○○○○○○○○○○○○○○○○○○○○○
○○○○○○○○○○○○○○○○○○○○○○○○○○○○○○○
○○○○○○○○○○○○○○○○○○○○○○○○○○○
 ○○○○○○○○○○○○○○○○○○○○○○○○○○
○○○○○○○○○○○○○○○○○○○○○○○○○○○○○○○○
%  ○○○○○○○○○○○○○○○○○○○○○○○○○○○○○○○
% ○○○○○○○○○○○○○○○○○○○○○○○○○○○
%  ○○○○○○○○○○○○○○○○○○○○○○○○○○
% ○○○○○○○○○○○○○○○○○○○○○○○○○○○○○○○○
%  ○○○○○○○○○

\vskip10Q
\noindent
\textbf{2.主結果}

ここでは,卒業論文の主結果を簡潔に説明してください.
必要な用語は(できれば)簡単に説明するなど,専門外の人にもわかるように
書いてください.
○○○○○○○○○○○○○○○○○○○○○○○○○○○○○○○○○○○
○○○○○○○○○○○○○○○○○○○○○○
○○○○○○○○○○○○○○○○○○○○○○
○○○○○○○○○○○○○○○○○○○○○○○○○○○○○○○○○○○
○○○○○○○○○○○○○○○○○○○○○○○○○○○○○○○○○○○
○○○○○○○○○○○○○○○○○○○○○○
○○○○○○○○○○○○○○○○○○○○○○
○○○○○○○○○○○○○○○○○○○○○○○○○○○○○○○○○○○
○○○○○○○○○○○○○○○○○○○○○○○○○○○○○○○○○○○
○○○○○○○○○○○○○○○○○○○○○○
○○○○○○○○○○○○○○○○○○○○○○
○○○○○○○○○○○○○○○○○○○○○○○○○○○○○○○○○○○
 ○○○○○○○○○○○○○○○○○○○○○○○○○○○○○○○○○○
○○○○○○○○○○○○○○○○○○○○○○○
% ○○○○○○○○○○○○○○○○○○○○○○
%○○○○○○○○○○○○○○○○○○○○○○○○○○○○○○○○○○○
% ○○○○○○○○○○○○○○○○○○○○○○○○○○○○○○○○○○
%○○○○○○○○○○○○○

\vskip10Q
\noindent
\textbf{3.意義・証明のアイデアや方法}

主結果の意義,証明(アイデアや方法)を説明してください.
とくに,新しいアイデアや,証明のポイントを簡潔に説明してください.
また,得られた主結果にどのような意義があるのか,専門外の人にわかるように,
明確に説明してください.
○○○○○○○○○○○○○○○○○○○○○○○○○○○○○○○○○○○○○
○○○○○○○○○○○○○○○○○
○○○○○○○○○○○○○○○○○
○○○○○○○○○○○○○○○○○○○○○○○○○○○○○○○○○○○○○
○○○○○○○○○○○○○○○○○○○○○○○○○○○○○○○○○○○○○
○○○○○○○○○○○○○○○○○
○○○○○○○○○○○○○○○○○
○○○○○○○○○○○○○○○○○○○○○○○○○○○○○○○○○○○○○
○○○○○○○○○○○○○○○○○○○○○○○○○○○○○○○○○○○○○
○○○○○○○○○○○○○○○○○
○○○○○○○○○○○○○○○○○
○○○○○○○○○○○○○○○○○○○○○○○○○○○○○○○○○○○○○
○○○○○○○○○○○○○○○○○○○○○○○○○○○○○○○○○○○○○
○○○○○○○○○○○○○○○○○
 ○○○○○○○○○○○○○○○○
○○○○○○○○○○○○○○○○○○○○○○○○○○○○○○○○○○○○○○
%  ○○○○○○○○○○○○○○○○○○○○○○○○○○○○○○○○○○○○○
% ○○○○○○○○○○○○○○○○○
%  ○○○○○○○○○○○○○○○○
% ○○○○○○○○○○○○○○○○○○○○○○○○○○○○

\vskip10Q
\noindent
\textbf{4.今後の課題}

今後,研究を続けるとしたら,どのような課題があるのか,
また,どのような問題を研究したいか,説明してください.
さらに,研究を発展させる計画や,予想などがあれば,
明確に説明してください.
○○○○○○○○○○○○○○○○○○○○○○○○○○○
○○○○○○○○○○○○○○○○○○○○○○○○○○○
○○○○○○○○○○○○○○○○○○○○○○○○○○○
○○○○○○○○○○○○○○○○○○○○○○○○○○○
○○○○○○○○○○○○○○○○○○○○○○○○○○○
○○○○○○○○○○○○○○○○○○○○○○○○○○○
○○○○○○○○○○○○○○○○○○○○○○○○○○○
○○○○○○○○○○○○○○○○○○○○○○○○○○○
○○○○○○○○

\vskip10Q
\noindent
\textbf{参考文献}

{\small

\noindent
[1] 著者名,論文題目,雑誌名,巻,号 (年号), ページ番号.

\noindent
[2] 著者名,論文題目,雑誌名,巻,号 (年号), ページ番号.


}

\end{document}
