\documentclass[twocolumn]{ujarticle}

\setlength{\topmargin}{-1.4cm}
\setlength{\oddsidemargin}{-30Q}
\setlength{\evensidemargin}{0cm}

\setlength{\columnsep}{2.5zw}
\usepackage[dvipdfmx, hidelinks]{hyperref}
\usepackage{amsmath}
\usepackage{amssymb}
\usepackage{amsthm}
\usepackage{eucal}

\newcommand{\ZZ}{\mathbb{Z}}
\newcommand{\QQ}{\mathbb{Q}}
\newcommand{\RR}{\mathbb{R}}
\newcommand{\NN}{\mathbb{N}}


\newcommand{\TITLE}
{Magnitude Homology of Graphs and the Magnitude as its Categorification}
\newcommand{\STNO}
{2264257}
\newcommand{\NAME}
{谷内 賢翔}
\newcommand{\ADVR}
{野崎 雄太 准教授}
\newcommand{\DATE}
{(2026年2月6日)}
\newcommand{\NENDO}
{2025年度}

\setlength{\textwidth}{52zw}
\setlength{\textheight}{25cm}


\makeatletter

\def\ps@lrheadf{\ps@empty
  \def\@evenhead{\normalfont
      \leftmark{}{}\hfil \rightmark{}{}}
  \let\@oddhead\@evenhead
  \def\@oddfoot{}
  \let\@evenfoot\@oddfoot
  \let\@mkboth\markboth}
\makeatother


\pagestyle{lrheadf}

\markboth{
数物・電子情報系学科数理科学EP 
\hskip1zw
卒業研究 \hskip1zw 概要
}{}


\begin{document}
\fontsize{10pt}{14pt}\selectfont


\twocolumn[
\centering
\textbf{\Large \TITLE}

\vskip5Q
\hfil 
{\large \STNO \hskip1zw \NAME
\hskip 2zw Supervisor : \ADVR}

\vskip20Q

]

\noindent
\textbf{1.研究背景・動機}

The concept of magnitude is introduced by Leinster and it is defined for enriched categories of finite objects, for example, generalized finite metric spaces such as finite graphs.
Then, Leinster focuses on the magnitude of graphs in using his idea of magnitude of a metric space, which is one of a family of cardinality-like invariants extending across mathematics;
it is a cousin to Euler characteristic and geometric measure. 
Among its cardinality-like properties are multiplicativity with respect to cartesian product and an inclusion-exclusion formula for the magnitude of a union under mild hypotheses. Formally, the magnitude of a graph is both a rational function over $\QQ$ and a power series over $\ZZ$.

Richard and Simon introduced a bigraded homology theory for graphs which has the magnitude as its graded Euler characteristic and showed how properties of magnitude proved by Leinster categorify to properties such as a Kunneth Theorem and a Mayer-Vietoris Theorem. 

Here, we first review the definition of the magnitude of graphs, the magnitude homology of graphs, and their properties.
Then we focus on the magnitude of enriched categories and discuss how the magnitude of graphs is introduced from that of enriched categories. We denote the magnitude of a graph $G$ by $\# G$ and the magnitude homology of $G$ by $\mathrm{MH}_{*,*}(G)$.

\vskip10Q
\noindent
\textbf{2.主結果}

The first main result is a certain property that we would like graph invariants to satisfy, the inclusion-exclusion formula;
\[
  \# (G \cup H) = \# G + \# H - \#(G \cap H).
\]
For this we must impose some hypotheses. Indeed, Leinster shows that there is no nontrivial graph invariant that is fully cardinality-like in the sense of satisfying both multiplication and inclusion-exclusion formula without restriction.
However, the hypotheses we impose are mild enough to cover a wide range of examples, including trees, forests, wedge sums, and graphs containing a cycle of length at least 4. For example, let $G$ be a graph shown below. 

The second main result is confirming that the magnitude defined in the context of enriched categories coincides with that defined in the context of graphs.

$ \# G$はmonoidal categoryで$\mid x \mid = e^x = q$としたものなど書く.

\vskip10Q
\noindent
\textbf{3.意義・証明のアイデアや方法}

We use the usage of graph theory.

\vskip10Q
\noindent
\textbf{4.今後の課題}

下記の中から未解決なものを残す. (torsionは解決済みなので消す).
There are examples of non-isomorphic graphs with isomorphic magnitude homol-
ogy, for example any two trees with the same number of vertices. Are there
graphs with the same magnitude but different magnitude homology groups?
• Is there a graph whose magnitude homology contains torsion?
• Leinster showed that if two graphs differ by a Whitney twist with adjacent gluing
points, then their magnitudes are equal. Do two graphs related by a Whitney
twist have isomorphic magnitude homology?
• Prove the magnitude homology of cyclic graphs is as is conjectured in Appen-
dix A.1.
• Computations suggest that the icosahedral graph (i.e., the 1-skeleton of the
icosahedron) has diagonal homology. We have not been able to apply any of our
techniques for proving that graphs are diagonal in this case, and, in particular,
the graph is not a join. Is the icosahedral graph diagonal, and if so why? More
generally, is there a graph-theoretic characterization of diagonal graphs?
• We anticipate that there is a theory of magnitude cohomology dual to the
homological theory developed in this paper. As with cohomology of spaces, it
should be possible to equip this theory with a product structure
• One can define MH∗,∗(G) as the reduced homology of a sequence of pointed
simplicial sets. This is used in Section 8, see, in particular, Remark 8.6. We have
chosen not to emphasise this approach in the present paper, but there may be
advantages to doing so in future.

\vskip10Q
\noindent
\textbf{参考文献}

{\small

\noindent
[1] 著者名,論文題目,雑誌名,巻,号 (年号), ページ番号.

\noindent
[2] 著者名,論文題目,雑誌名,巻,号 (年号), ページ番号.


}

\end{document}
