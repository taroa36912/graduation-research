\documentclass[dvipdfmx]{standalone}
\usepackage{tikz}
\usetikzlibrary{calc} % 座標計算のために必要

\begin{document}

\begin{tikzpicture}
    % ===================================
    % 設定
    % ===================================
    \def\R{2.0} % 半径
    \tikzset{dot/.style={circle, fill=black, inner sep=2pt}}
    % ===================================

    % 1. 頂点配置
    \node[dot] (v1) at (120:\R) {}; % 左上
    \node[dot] (v2) at (60:\R) {};  % 右上
    \node[dot] (vn) at (300:\R) {}; % 右下

    % 2. 仮想的な隣の頂点(描画はしない、方向の基準用)
    \coordinate (next_to_v1) at (180:\R);
    \coordinate (next_to_v2) at (0:\R);
    \coordinate (next_to_vn_1) at (0:\R);
    \coordinate (next_to_vn_2) at (240:\R);

    % 3. 完全グラフの辺(3頂点を互いに結ぶ)
    \draw (v1) -- (v2);
    \draw (v2) -- (vn);
    \draw (vn) -- (v1);

    % 4. 正六角形の輪郭を「少しだけ」伸ばす
    % ($(A)!0.3!(B)$) は「AからBに向かって30%の位置」という意味です
    \draw (v1) -- ($(v1)!0.3!(next_to_v1)$);
    \draw (v2) -- ($(v2)!0.3!(next_to_v2)$);      % 右上から右へ
    \draw (vn) -- ($(vn)!0.3!(next_to_vn_1)$);    % 右下から右へ
    \draw (vn) -- ($(vn)!0.3!(next_to_vn_2)$);

    % 5. 省略記号 (...) の配置
    % 右側:右へ伸びる2本の辺のちょうど間(0度地点)
    \node at (0:\R) {$\dots$};

    % 左下側:斜めに配置
    \node[rotate=60] at (210:\R) {$\dots$};

\end{tikzpicture}

\end{document}