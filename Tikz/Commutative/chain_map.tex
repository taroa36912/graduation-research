\documentclass[dvipdfmx, border=10pt]{standalone}
% standaloneクラス: 描画された図のサイズに合わせてPDFのページサイズを自動調整します。
% border=10pt: 図の周りに少し余白を持たせます(窮屈にならないように)。

\usepackage{amsmath}
\usepackage{amssymb}
\usepackage{tikz-cd} % 可換図式用のパッケージ

\begin{document}

% 図式の定義
\begin{tikzcd}[row sep=large, column sep=large]
    % --- 1行目 ---
    \cdots \arrow[r] 
    & MC_{k,l}(G) \arrow[r, "\partial"] \arrow[d, "f_{\#}^{k,l}"'] 
    & MC_{k-1, l}(G) \arrow[r] \arrow[d, "f_{\#}^{k-1,l}"'] 
      % Bから左下(C)に向かって透明な矢印を引き、そこに回転矢印を置く
      \arrow[dl, phantom, "\circlearrowright"]
    & \cdots \\
    % --- 2行目 ---
    \cdots \arrow[r] 
    & MC_{k, l}(H) \arrow[r, "\partial"'] 
    & MC_{k-1, l}(H) \arrow[r] 
    & \cdots
\end{tikzcd}

\end{document}