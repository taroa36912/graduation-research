\documentclass[12pt]{ujarticle}
\renewcommand{\abstractname}{Abstract}
\renewcommand{\contentsname}{Contents}
\renewcommand{\refname}{References}

\usepackage[dvipdfmx, hidelinks]{hyperref}
\usepackage{amsmath}
\usepackage{amssymb}

\usepackage{amsthm}
\theoremstyle{plain}
\newtheorem{theorem}{Theorem}[subsection]
\newtheorem{proposition}{Proposition}[subsection]
\newtheorem{corollary}{Corollary}[subsection]
\theoremstyle{definition}
\newtheorem{definition}{Definition}[subsection]
\newtheorem{example}{Example}[subsection]
\newtheorem{remark}{Remark}[subsection]

\setlength{\topmargin}{0truecm}

\newcommand{\TITLE}%
{Magnitude Homology of Graphs and the Magnitude as its Categorification}
%上のタイトルを書く
\newcommand{\STNO}%
{2264257}%学籍番号
\newcommand{\NAME}%
{Kensho Yachi}%氏名
\newcommand{\ADVR}%
{Yuta Nozaki Associate Professor}%指導教員

\newcommand{\DATE}%
{(January 30th, 2025)}%〆切




\begin{document}
\thispagestyle{empty}

\begin{center}
  2025 Yokohama National University, Faculty of Science and Engineering, Mathematical Science EP Graduation Research
  \vskip3cm
\end{center}



{\Large

\begin{center}
  %\LARGE
  \huge
  \textbf{\TITLE}
\end{center}

\vfill

\hfil
{\LARGE
\textbf{
\STNO\ \NAME}
}

\vskip 4mm
\hfil
{\large
\href{https://taro-ken.com}{https://taro-ken.com}
}

\vskip10Q
\begin{center}
  \textbf{Supervisor : \ADVR }
\end{center}


\hfil
\textbf{\DATE}

}


\vfill
\hfill
\begin{tabular}{|p{6zw}|p{6zw}|} \hline
   \hskip.5zw 指導教員印 & \hskip1.5zw 受理印 \\ \hline
   & \\[2cm] \hline 
 
\end{tabular}



% Start of the main content
\newpage
\pagestyle{plain}
\setcounter{page}{1}


% Abstract
\begin{abstract}
Sample Abstract
\end{abstract}


% Table of Contents
\newpage
\tableofcontents


% Section 1
\newpage
\section{Introduction}
Lamport's guide to \LaTeX\ \cite{Lamport94}.


% Section 2
\newpage
\section{The definition}
In this section, we define the magnitude of a graph $G$ and the magnitude homology of a graph $G$, give some very basic examples and properties.
By a \textit{graph} we mean a finite undirected graph with no loops or multiple edges. The set of vertices of a graph $G$ is denoted by $V(G)$, and the set of edges of $G$ is denoted by $E(G)$. If $x$ and $y$ are vertices of a graph $G$, then the \textit{distance} $d_G(x,y)$ between $x$ and $y$ is defined to be the length of a shortest edge path from $x$ to $y$. If $x$ and $y$ lie in different components of $G$ then $d(x,y)=\infty$.
\subsection{The Definition of The Magnitude and Some Examples, Properties}
Here, we difine the magnitude of a graph, which can be expressed as either a rational function over $\mathbb{Q}$ or a formal power series over $\mathbb{Z}$. Write $\mathbb{Z}[q]$ for the polynomial ring over the integers in one variable $q$.

\begin{definition}
  Let $G$ be a graph. Let $Z_G = Z_G(q)$ be the square matrix over $\mathbb{Z}[q]$ whose rows and columns are indexed by the vertices of $G$, and whose $(x,y)$-entry is given by
  \[
  Z_G(q)(x,y) = q^{d(x,y)} \quad (x,y \in V(G))
  \]
  where by convention $q^\infty = 0$.
\end{definition}

\subsection{The Definition of The Magnitude Homology and Some Examples, Properties}
Magnitude homology



% Bibliography (参考文献)
\newpage
\begin{thebibliography}{9}

\bibitem{Lamport94}
Leslie Lamport.
\textit{LaTeX: A Document Preparation System}.
Addison-Wesley, 2nd edition, 1994.

\bibitem{Knuth84}
Donald E. Knuth.
\textit{The TeXbook}.
Addison-Wesley, 1984.

\end{thebibliography}



\end{document}
