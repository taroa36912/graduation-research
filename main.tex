\documentclass[12pt]{ujarticle}
\renewcommand{\abstractname}{Abstract}
\renewcommand{\contentsname}{Contents}
\renewcommand{\refname}{References}

\usepackage{amsmath}
\usepackage{amsthm}
\usepackage{aliascnt}
\usepackage[dvipdfmx, hidelinks]{hyperref}
\usepackage[capitalize]{cleveref}
\usepackage[dvipdfmx]{graphicx}
\usepackage{amssymb}
\usepackage{bm}
\usepackage{enumitem}
\usepackage{stmaryrd}
\usepackage{chngcntr}

\theoremstyle{plain}
\newtheorem{theorem}{Theorem}[subsection]
\crefname{theorem}{Theorem}{Theorems}
\Crefname{theorem}{Theorem}{Theorems}
\newaliascnt{proposition}{theorem}
\newtheorem{proposition}[proposition]{Proposition}
\aliascntresetthe{proposition}
\crefname{proposition}{Proposition}{Propositions}
\Crefname{proposition}{Proposition}{Propositions}
\newaliascnt{lemma}{theorem}
\newtheorem{lemma}[lemma]{Lemma}
\aliascntresetthe{lemma}
\crefname{lemma}{Lemma}{Lemmas}
\Crefname{lemma}{Lemma}{Lemmas}
\newaliascnt{corollary}{theorem}
\newtheorem{corollary}[corollary]{Corollary}
\aliascntresetthe{corollary}
\crefname{corollary}{Corollary}{Corollaries}
\Crefname{corollary}{Corollary}{Corollaries}
\theoremstyle{definition}
\newaliascnt{definition}{theorem}
\newtheorem{definition}[definition]{Definition}
\aliascntresetthe{definition}
\crefname{definition}{Definition}{Definitions}
\Crefname{definition}{Definition}{Definitions}
\newaliascnt{example}{theorem}
\newtheorem{example}[example]{Example}
\aliascntresetthe{example}
\crefname{example}{Example}{Examples}
\Crefname{example}{Example}{Examples}
\newaliascnt{remark}{theorem}
\newtheorem{remark}[remark]{Remark}
\aliascntresetthe{remark}
\crefname{remark}{Remark}{Remarks}
\Crefname{remark}{Remark}{Remarks}

\setlength{\topmargin}{0truecm}

\newcommand{\TITLE}%
{Magnitude Homology of Graphs and the Magnitude as its Categorification}
%上のタイトルを書く
\newcommand{\STNO}%
{2264257}%学籍番号
\newcommand{\NAME}%
{Kensho Yachi}%氏名
\newcommand{\ADVR}%
{Yuta Nozaki Associate Professor}%指導教員

\newcommand{\DATE}%
{(January 30th, 2025)}%〆切




% 汎用コマンド
\newcommand{\tsum}{\text{sum}}
\newcommand{\tadj}{\text{adj}}
\newcommand{\aut}{\text{Aut}}
\newcommand{\isom}{\xrightarrow{\sim}}
\newcommand{\mor}{\text{Mor}}
\newcommand{\rank}{\text{rank}}
\newcommand{\sign}{\text{sign}}
\newcommand{\tor}{\mathrm{Tor}}
\newcommand{\Ker}{\mathrm{Ker}}
\newcommand{\im}{\mathrm{Im}}
\newcommand{\ZZ}{\mathbb{Z}}
\newcommand{\QQ}{\mathbb{Q}}
\newcommand{\NN}{\mathbb{N}}
\newcommand{\RR}{\mathbb{R}}
\newcommand{\CC}{\mathcal{C}}
\newcommand{\VV}{\mathcal{V}}
\newcommand{\calA}{\mathcal{A}}
\newcommand{\graph}[1]{
  \raisebox{-0.5\height}{\includegraphics{Tikz/Graphs/out/#1}}%
}



\begin{document}
\thispagestyle{empty}

\begin{center}
  2025 Yokohama National University, Faculty of Science and Engineering, Mathematical Science EP Graduation Research
  \vskip3cm
\end{center}



{\Large

\begin{center}
  %\LARGE
  \huge
  \textbf{\TITLE}
\end{center}

\vfill

\hfil
{\LARGE
\textbf{
\STNO\ \NAME}
}

\vskip 4mm
\hfil
{\large
\href{https://taro-ken.com}{https://taro-ken.com}
}

\vskip10Q
\begin{center}
  \textbf{Supervisor : \ADVR }
\end{center}


\hfil
\textbf{\DATE}

}


\vfill
\hfill
\begin{tabular}{|p{6zw}|p{6zw}|} \hline
   \hskip.5zw Supervisor's seal & \hskip1.5zw acceptance stamp \\ \hline
   & \\[2cm] \hline 
 
\end{tabular}



% Start of the main content
\newpage
\pagestyle{plain}
\setcounter{page}{1}


% Abstract
\begin{abstract}
Sample Abstract
\end{abstract}


% Table of Contents
\newpage
\tableofcontents


% Section 1
\newpage
\section{Introduction}
Lamport's guide to \LaTeX\ \cite{Lamport94}.
We denote $\approx$ as isomorphisms and $\cong$ as homeomorphisms.


% Section 2
\newpage
\section{The Magnitude of Graphs}
This section introduces the magnitude and the magnitude homology of a graph $G$, along with fundamental examples and properties.
Throughout this paper, a \textit{graph} means a finite undirected graph with no loops or multiple edges. The set of vertices of a graph $G$ is denoted by $V(G)$, and the set of edges of $G$ is denoted by $E(G)$. For vertices $x, y \in V(G)$, the \textit{distance} $d_G(x,y)$ is defined as the length of a shortest edge path between them. If $x$ and $y$ lie in different connected components of $G$, we set $d_G(x,y)=\infty$.

\subsection{The Definition of the Magnitude of Graphs}
We begin by defining the magnitude of a graph. This invariant takes values in either the field of rational functions over $\QQ$ or the ring of formal power series over $\ZZ$.
Let $\QQ(q)$ denote the field of rational functions in a variable $q$ over $\QQ$. We also write $\ZZ[q]$ and $\ZZ \llbracket q \rrbracket$ for the polynomial ring and the ring of formal power series in $q$ over $\ZZ$, respectively.

\begin{definition}
  Let $G$ be a graph. We define the \textit{$G$-matrix} $Z_G = Z_G(q)$ over $\ZZ[q]$ whose rows and columns are indexed by the vertices of $G$, and whose $(x,y)$-entry is given by
  \[
  Z_G(q)(x,y) = q^{d(x,y)} \quad (x,y \in V(G))
  \]
  where by convention $q^\infty = 0$.
  
\end{definition}

$G$-matrix is the square synmetric matrix.

\begin{proposition}\label{invertible}
  A の$G$-matrix is invertible.
\end{proposition}

\begin{proof}
  By definition, the determinant of $Z_G$ has constant term $1$, which implies that $\det Z_G \neq 0$.
\end{proof}

\begin{definition}
  The \textit{magnitude} of a graph $G$ is defined to be the rational function given by
  \[
  \# G(q) = \sum_{x,y \in V(G)} (Z_G(q))^{-1}(x,y).
  \]
\end{definition}

\begin{remark}
  \[
    \# G(q) = \tsum(Z_G(q)^{-1}) = \frac{\tsum(\tadj(Z_G(q)))}{\det(Z_G(q))},
  \]
  where $\tadj$ is the adjugate matrix and $\tsum$ is the sum of all entries of a matrix.
\end{remark}

\begin{proposition}
  $\# G(q)$ takes values in $\ZZ \llbracket q \rrbracket$.
\end{proposition}

\begin{proof}
  Let $\det Z_G(q) = 1 - qf(q)$ for some $f(q) \in \ZZ[q]$ by \cref{invertible}. Then we have 
  \[
    \# G(q) = \frac{\tsum(\tadj(Z_G))}{\det(Z_G)} = \tsum(\tadj(Z_G)) \sum_{n=0}^{\infty} q^n f(q)^n.
  \]
  Note that $qf(q)$ has no constant term and then $\sum_{n=0}^{\infty} q^n f(q)^n$ takes values in $\ZZ \llbracket q \rrbracket$.
\end{proof}

\begin{example}
  Let $G = K_3$ (complete graph with three vertices). \\
  \graph{K_3.pdf} \\
  \\
  Then, we can calculate the magnitude of $K_3$ as follows:
  \[
    Z_{K_3}(q) = \begin{pmatrix}
      1 & q & q \\
      q & 1 & q \\
      q & q & 1
    \end{pmatrix}, \quad
    Z_{K_3}(q)^{-1} = \frac{1}{1 - 3q^2 + 2q^3} \begin{pmatrix}
      1 - q^2 & -q + q^2 & -q + q^2 \\
      -q + q^2 & 1 - q^2 & -q + q^2 \\
      -q + q^2 & -q + q^2 & 1 - q^2
    \end{pmatrix},
  \]
  $ \#K_3(q) = \frac{3}{1+2q}.$

\end{example}

\begin{definition}
  Let $G$ be a graph and $x \in V(G)$.
  The \textit{weight} of $x$ in $G$ is defined 
  \[
    w_G(x)(q) = \sum_{y \in V(G)}(Z_G(q))^{-1}(x,y)
  \]

  The function $w_G : V(G) \rightarrow \QQ(q)$ is called the \textit{weighting} on $G$.
\end{definition}

The magnitude can be expressed using the weighting as follows:
\[
  \# G(q) = \sum_{x \in V(G)} w_G(x)
\]

\begin{lemma}\label{weighting equation}
  For any graph $G$, the weighting $w_G$ satisfies
  \[
    \sum_{y \in V(G)} q^{d(x,y)} w_G(y) = 1 \quad (x \in V(G)).
  \]
\end{lemma}

\begin{proof}
  For any vertex $x \in V(G)$, we have 
  \[
  \begin{split}
    \sum_{y \in V(G)}q^{d(x,y)} w_G(y) &= \sum_{y,z \in V(G)}q^{d(x,y)}Z_G^{-1}(y,z) \\
    &= \sum_{y,z \in V(G)}Z_G(x,y)Z_G^{-1}(y,z) \\
    &= \sum_{z \in V(G)} \sum_{y \in V(G)}Z_G(x,y)Z_G^{-1}(y,z) \\
    &= \sum_{z \in V(G)}(Z_G Z_G^{-1})(x,z) \\
    &= \sum_{z \in V(G)}I(x,z) \\
    &= 1.
  \end{split}
  \]
\end{proof}

This equation is called the \textit{weighting equation}.

\begin{lemma}\label{unique property on weighting equation}
  Let $G$ be a graph and $\tilde{w}_G : V(G) \rightarrow \QQ$ be a function satisfying a weighting equation.
  Then, $\tilde{w}_G = w_G$. Now, $w_G$ is the weighting on $G$.
\end{lemma}

\begin{proof}
  Let $\bm{b} = (1, 1, \ldots, 1)^T$ where the length of $\bm{b}$ is $|V(G)|$ and $\bm{w}_G$ = $(w_G(x))^T_{x \in V(G)}$.
  If $\tilde{w}_G$ satisfies the weighting equation, then we have
  \[
    Z_G \tilde{\bm{w}}_G = \bm{b}.
  \]
  Since $Z_G$ is invertible by \cref{invertible}, we have $\tilde{w}_G = w_G$.
\end{proof}

This lemma shows that the weighting on a graph is unique and we use this frequently to compute the magnitude of graphs.


\subsection{Basic Properties and Examples}
This subsection presents fundamental properties and examples of magnitude. We focus on vertex-transitive graphs, disjoint unions, and Cartesian products. We also discuss the properties of magnitude within $\ZZ \llbracket q \rrbracket$.

\begin{definition}
  Let $G = (V(G), E(G)), H = (V(H), E(H))$ be graphs.
  A \textit{graph homomorphism} from $G$ to $H$ is a map $f : V(G) \rightarrow V(H)$ such that if $\{x,y\} \in E(G)$ then $\{f(x), f(y)\} \in E(H)$ or $f(x) = f(y)$.
\end{definition}

We can define a \textit{graph automorphism} using the definition above. We denote the group of all graph automorphisms of a graph $G$ by $\aut(G)$. $\aut(G)$ includes $id_G$ and for $g,h \in \aut(G)$ and $x \in V(G)$, $g(h(x)) = (gh)(x)$, which means $\aut(G)$ acts on $V(G)$.
\begin{definition}
  A graph $G$ is \textit{vertex-transitive} if $\aut(G)$ acts transitively on $V(G)$.
  It says that for any vertices $x$ and $y$ of $G$, there exists an automorphism $g : G \rightarrow G$ such that $y = g(x)$.
\end{definition}

\begin{lemma}
  Let $G$ be a vertex-transitive graph.
  Then, 

  \[
    \# G(q) = \frac{|V(G)|}{\sum_{y \in V(G)} q^{d(x,y)}}
  \]

  for any vertex $x \in V(G)$.
\end{lemma}

\begin{proof}
  Let $S(x) = \sum_{y \in V(G)} q^{d(x,y)}$ for a vertex $x \in V(G)$. We show that $S(x)$ does not depend on the choice of $x$.  Take any vertices $a,b \in V(G)$.
  Since $G$ is vertex-transitive, there exists $g \in \aut(G)$ such that $b = g(a)$.
  Then,
  \[
    \begin{split}
      S(b) &= \sum_{y \in V(G)} q^{d(b,y)} \\
      &= \sum_{y \in V(G)} q^{d(g(a),y)} \\
      &= \sum_{y \in V(G)} q^{d(g(a), g(y))} \quad (g \text{ is bijective}) \\
      &= \sum_{y \in V(G)} q^{d(a,y)} \quad (g \text{ is an isomorphism})  \\
      &= S(a).
    \end{split}
  \]

  Thus, $S(x)$ does not depend on the choice of $x$, denoting it by $S$.
  Now, we define a function $\tilde{w}_G : V(G) \rightarrow \QQ(q)$ by $\tilde{w}_G(x) = \frac{1}{S}$ for any vertex $x \in V(G)$.
  Then $\tilde{w}_G$ satisfies the weighting equation and by \cref{unique property on weighting equation}, we have $\# G = \frac{|V(G)|}{S}$.
  
\end{proof}


\begin{example}
  \begin{enumerate}[label=(\roman*)]
    \item $G = V_n$ (edgeless graph with $n$ vertices). \\
    \graph{n_vertex.pdf} \\ 
    Then, $\aut(G) \approx \mathfrak{S}_n$ and $G$ is vertex-transitive. $S = 1$ and we have $\# V_n = n$.
    \item $G = K_n$ (complete graph with $n$ vertices). \\
    \graph{K_n.pdf} \\
    Then, $\aut(G) \approx \mathfrak{S}_n$ and $G$ is vertex-transitive. $S = 1 + (n-1)q$ and we have $\# K_n = \frac{n}{1 + (n-1)q}$.
    \item $G = C_n$ (cycle graph with $n$ vertices). \\
    \graph{C_n.pdf} \\
    Then, $\aut(G) \approx D_{2n}$ and $G$ is vertex-transitive.
    If $n = 2m$, then $S = 1 + 2(q + q^2 + \cdots + q^{m-1}) + q^m = \frac{1 + q - q^m - q^{m+1}}{1 - q}$. Thus, we have $\# C_{2m} = \frac{2m(1-q)}{(1+q)(1-q^m)} = \frac{n(1-q)}{(1+q)(1-q^m)}$.
    If $n=2m-1$, then similarly 
    $\# C_{2m-1} = \frac{n(1-q)}{1+q-2q^m}$.
    \item $G$ is a Petersen graph. \\
    \graph{Petersen.pdf} \\
    Then, $\aut(G)$ contains $D_{10}$ as its subgroup and $G$ is vertex-transitive. $S = 1 + 3q + 6q^2$ and we have $\# G = \frac{10}{1 + 3q + 6q^2}$.
    \item $G = K_{m,n}$(complete bipartite graph). \\
    \graph{K_mn.pdf} \\
    Then, $\aut(G) \approx \mathfrak{S}_m \times \mathfrak{S}_n$ if $m \neq n$ and $G$ is not vertex-transitive. We can calculate the magnitude with other methods. Let $a,b$ be the weight of vertices on each part of $K_{m,n}$. Then, the weighting equation is written by two equations as follows:
    \[
      \begin{cases}
        \{ q^0 + (m-1)q^2 \} a + nqb = 1 \\
        \{q^0 + (n-1)q^2 \} b + mqa = 1.
      \end{cases}
    \]
    We can solve this and we have 
    \[
      \# K_{m,n} = ma + nb = \frac{(m+n) - (2mn - m - n)q}{(1+q)(1-(m-1)(n-1)q^2)}.
    \]
  \end{enumerate}
\end{example}

\begin{lemma}\label{disjoint_union}
  Let $G$ and $H$ be graphs. Then, 
  \[
    \#(G \sqcup H) = \# G + \# H,
  \]
  where $G \sqcup H$ is the disjoint union of $G$ and $H$.
\end{lemma}

\begin{proof}
  $Z_{G \sqcup H} = 
  \begin{pmatrix}
    Z_G & O \\
    O & Z_H
  \end{pmatrix}, 
  Z_{G \sqcup H}^{-1} = 
  \begin{pmatrix}
    Z_G^{-1} & O \\
    O & Z_H^{-1}
  \end{pmatrix}$.

  Thus, 
  \[
    \#(G \sqcup H) = \tsum(Z_{G \sqcup H}^{-1}) = \tsum(Z_G^{-1}) + \tsum(Z_H^{-1}) = \# G + \# H.
  \]
\end{proof}

\begin{definition}
  Let $G$ and $H$ be graphs. The \textit{cartesian product} $G \square H$ of $G$ and $H$ is the graph defined as follows;
  \begin{itemize}
    \item $V(G \square H) = V(G) \times V(H)$.
    \item $E(G \square H) = \{\{ (x,y), (x', y') \} | x=x' \text{ and } \{y,y'\} \in E(H) \text{ or } y=y' \text{ and } \{x,x'\} \in E(G)\}$.
  \end{itemize}
\end{definition}

\begin{lemma}
  $\# G \square H = \# G \cdot \# H$.
\end{lemma}

\begin{proof}
  For $x, x' \in V(G)$ and $y, y' \in V(H)$, \\
  $d_{G \square H}((x,y), (x',y')) = d_G(x,x') + d_H(y,y')$, \\
  $Z_{G \square H}((x,y), (x',y')) = q^{d_{G \square H}((x,y), (x',y'))} = q^{d_G(x,x')} q^{d_H(y,y')} = Z_G(x,x') Z_H(y,y')$, \\
  $Z_{G \square H} = Z_G \otimes Z_H$ and then $Z_{G \square H}^{-1} = Z_G^{-1} \otimes Z_H^{-1}$. \\
  Thus, 
  $\# G \square H = \tsum(Z_{G \square H}^{-1}) = \tsum(Z_G^{-1} \otimes Z_H^{-1}) = \tsum(Z_G^{-1}) \cdot \tsum(Z_H^{-1}) = \# G \cdot \# H$.\\

  We used the fact that $(P \otimes Q)(R \otimes S) = (PR) \otimes (QS)$ for proper matrices $P,Q,R, \text{ and } S$.
\end{proof}

\begin{example}
  $G = K_2 \square K_3$. \\
  $\# K_2 \square K_3 = \# K_2 \cdot \# K_3 = \frac{2}{1+q} \cdot \frac{3}{1+2q} = \frac{6}{(1+q)(1+2q)} = \# K_{3,3}$.
\end{example}

\begin{remark}
  Here we use the catesian product for graph product, but there are other graph products. However, there is a reason that we use the catesian product. This will be clear in Section 4.
\end{remark}

\begin{proposition}\label{sum_of_magnitude}
  Let $G$ be a graph. Then,
  \[
  \begin{split}
    \# G(q) &= \sum_{k=0}^{\infty} (-1)^k \sum_{x_0 \neq x_1 \neq \cdots \neq x_k} q^{d(x_0, x_1) + d(x_1, x_2) + \cdots + d(x_{k-1}, x_k)} \\
    &= \sum_{n=0}^{\infty} c_n q^n,
  \end{split}
  \]
  where 
  \[
    c_n = \sum_{k=0}^{n} (-1)^k |\{ (x_0, \dots, x_k) | x_0 \neq x_1 \neq \cdots \neq x_k, d(x_0, x_1) + \cdots + d(x_{k-1}, x_k) = n \}|.
  \]
\end{proposition}

\begin{proof}
  aaa.
\end{proof}

\begin{corollary}
  Let $G$ be a graph.
  $|V(G)| = \# G(0), |E(G)| = - \frac{1}{2} \left. \frac{d}{dq} \# G(q) \right|_{q=0}$.
  Here, the derivative is taken in $\ZZ \llbracket q \rrbracket$.
\end{corollary}

\begin{proof}
  From the previous proposition, we have
  \[
    \begin{split}
      c_0 &= \sum_{k=0}^{0}(-1)^k |\{ (x_0, \dots, x_k) | x_0 \neq x_1 \neq \cdots \neq x_k, d(x_0, x_1) + \cdots + d(x_{k-1}, x_k) = n \}| \\
      &= |\{ (x_0) | x_0 \in V(G) \}| \\
      &= |V(G)|
    \end{split}
  \]
  and
  \[
    \begin{split}
      c_1 &= |\{ (x_0) | d(x_0, x_0) = 1 \}| - |\{ (x_0, x_1) | x_0 \neq x_1, d(x_0, x_1) = 1 \}| \\
      &= 0 - 2|E(G)| \\
      &= -2|E(G)|.
    \end{split}
  \]
  This corollary immidiately follows from these equations.
\end{proof}

\begin{remark}
  $c_0 \geq 0, c_1 \leq 0, \text{ and } c_2 \geq 0$. $c_2 = 0$ if and only if
\end{remark}

\subsection{Main Results on the Magnitude of Graphs}
This subsection states the inclusion-exclusion principle for the magnitude of graphs under specific conditions. We begin by observing that the magnitude does not generally satisfy the inclusion-exclusion principle. We then introduce sufficient conditions for the principle to hold.
In this document, we mean $G \cup H$ as a graph $(V(G) \cup V(H), E(G) \cup E(H))$.

\begin{definition}
  Let $R$ be a ring. A function $\Phi$ is an \textit{$R$-valued graph invariant} if
  \begin{itemize}
    \item $\Phi(G) \in R$ for any graph $G$,
    \item If $G \approx H$ as a graph then $\Phi(G) = \Phi(H)$.
  \end{itemize}
\end{definition}

\begin{definition}
  Let $\Phi$ be an $R$-valued graph invariant. \\
  1. $\Phi$ is said to be \textit{multiplicative} if
  \begin{itemize}
    \item $\Phi(K_1) = 1$,
    \item $\Phi(G \square H) = \Phi(G) \cdot \Phi(H)$ for any graphs $G$ and $H$.
  \end{itemize}

  2. $\Phi$ is said to satisfy the \textit{inclusion-exclusion principle} if
  \begin{itemize}
    \item $\Phi(\emptyset) = 0$,
    \item $\Phi(G \cup H) = \Phi(G) + \Phi(H) - \Phi(G \cap H)$ for any graphs $G$ and $H$.
  \end{itemize}
\end{definition}

\begin{lemma}
  Let $R$ be a ring containing no nonzero nilpotents and $\Phi$ be a multiplicative $R$-valued graph invariant satisfying the inclusion-exclusion principle. Then, $\Phi(G) = |V(G)|$ for any graph $G$.
\end{lemma}

\begin{proof}
  aaa
\end{proof}

\begin{corollary}
  The magnitude does not satisfy the inclusion-exclusion principle in general.
\end{corollary}

\begin{example}label{Willerton}
  
\end{example}

\begin{definition}
  Let $X$ be a graph and $U$ be a subgraph of $X$.
  $U$ is said to be \textit{convex} in $X$ if for any vertices $x,y \in V(U)$, $d_U(x,y) = d_X(x,y)$.
\end{definition}

\begin{lemma}
  Let $X$ be a graph, $G,H$ be subgraphs of $X$ such that $X = G \cup H$, and $g \in V(G)$ and $h \in V(H)$ such that there is a path $(g = x_0 \rightarrow x_1 \rightarrow \dots \rightarrow x_n = h)$ in $X$.
  Then, there exists a vertex $x_i \in V(G) \cap V(H)$.
\end{lemma}

\begin{proof}
  aaa
\end{proof}

\begin{lemma}
  Let $X$ be a graph and $G,H$ be subgraphs of $X$ such that $X = G \cup H$.
  If $G \cap H$ is convex in $X$, then $G$ and $H$ are also convex in $X$.
\end{lemma}

\begin{proof}
  aaa
\end{proof}

\begin{definition}
  Let $X$ be a graph and $U$ be a subgraph of $X$ such that $U$ is convex in $X$. We denote $V_U(X) = \{ v \in V(X) | d_X(v,u) < \infty \text{ for some } u \in V(U) \}$.
  Then, we say that \textit{$X$ projects to $U$} if for any $x \in V_U(X)$, there exists $\pi(x) \in V(U)$ such that for any $u \in V(U)$, $d_X(x,u) = d_X(x, \pi(x)) + d_X(\pi(x),u)$.
\end{definition}

\begin{lemma}
  If $X$ projects to $U$, then $\pi(x)$ is uniquely determined for any $x \in V_U(X)$.
\end{lemma}

\begin{proof}
  aaa
\end{proof}

\begin{example}
  aaa
\end{example}

\begin{lemma}
  Let $X$ be a graph and $U \subset X$ be a convex subgraph of $X$ such that $X$ projects to $U$.
  Then, for any $u \in V(U)$, 
  \[
    w_U(u) = \sum_{x \in \pi^{-1}(u)}q^{d_X(u, x)}w_X(x).
  \]
\end{lemma}

\begin{proof}
  aaa
\end{proof}

\begin{theorem}\label{incl-excl}(Main theorem I)
  Let $X$ be a graph and $G,H$ be subgraphs of $X$ such that $X = G \cup H$.
  If $G \cap H$ is convex in $X$ and $H$ projects to $G \cap H$, then
  \[
    \# X = \# G + \# H - \#(G \cap H).
  \]
\end{theorem}

Before proving this theorem, we give the example of graphs for which we can apply this theorem.

\begin{example}
  Let  $G$ be a graph and consider the graph $H$ formed by identifying one of the edges of a cycle graph $C_n$ with an edge of $G$. Now, let $n \geq  4$. 

  \graph{connect_C_4.pdf}

  Then, we can apply \cref{incl-excl} to $X = G \cup H$ as follows:
  \[
    \# X = \# G + \# C_n - \# K_2.
  \]
  Similarly, if $G$ and $H$ are graphs and $G \vee H$ is the graph formed by identifying one vertex of $G$ with one vertex of $H$, then we have
  \[
    \#(G \vee H) = \# G + \# H - 1.
  \]
\end{example}


\begin{proof}[Proof of \cref{incl-excl}]
  aaa
\end{proof}

\begin{example}
  The three graphs below are divided into a graph $C_3$, and two graphs $C_2$, so they all have the same magnitude and can be calculated as follows:
  \[
    \# G = \# C_3 + 2 \cdot \# C_2 - 2.
  \]
\end{example}

\begin{example}
  If $G$ is a forest, then we can calculate the magnitude of $G$ as follows:
  \[
    \# G = |V(G)| - 2|E(G)| \frac{q}{1+q}.
  \]

  If $G$ is a tree, then
  \[
    \# G = |V(G)| - 2(|V(G)| - 1) \frac{q}{1+q}.
  \]
\end{example}

Furthermore examples.



% Section 3
\newpage
\section{The Magnitude Homology of Graphs}
In this section, we define the magnitude homology of a graph $G$. We provide fundamental examples and properties, and state the Mayer-Vietoris sequence for magnitude homology.

\subsection{The Definition of The Magnitude Homology of Graphs}
\begin{definition}
  Let $G$ be a graph. For positive integer $k$, the \textit{length} of a tuple $(x_0, \dots, x_k)$ of $V(G)$ is defined to be
  \[
    \begin{split}
      l(x_0, \dots, x_k) &= d(x_0, x_1) + d(x_1, x_2) + \cdots + d(x_{k-1}, x_k) \\
      &= \sum_{i=1}^{k} d(x_{i-1}, x_i).
    \end{split}
  \]
  Now, let $l(x_0) = 0$.
  We say the tuple $(x_0, \dots, x_k)$ is \textit{good} if $x_0 \neq x_1 \neq \cdots \neq x_k$.
\end{definition}

\begin{lemma}(Triangle inequality)
  If $(x_0, \dots, x_k)$ is a good tuple of $V(G)$, then for any $1 \leq i \leq k-1$,
  \[
    l(x_0, \dots, x_k) \geq l(x_0, \dots, \hat{x_i}, \dots, x_k).
  \]
\end{lemma}

\begin{proof}
  We obviously have the statement by the triangle inequality of the distance function $d$.
\end{proof}

\begin{definition}(\textit{magnitude chain complex})
  Let $G$ be a graph.
  $MC_{*,*}(G)$ is the \textit{magnitude complex} defined as follows:
  \[
    MC_{*,*}(G) = \bigoplus_{l=0}^{\infty} MC_{*, l}(G).
  \]
  For non-negative integers $k$ and $l$, $MC_{k,l}(G)$ is freely generated by good tuples $(x_0, \dots, x_k)$ of $V(G)$ of length $l$ with the ring $\ZZ$.
  The differential $\partial : MC_{k,l}(G) \rightarrow MC_{k-1,l}(G)$ is defined by
  \[
    \partial = \sum_{i=1}^{k-1} (-1)^{i-1} \partial_i,
  \]
  where 
  \[
    \partial_i (x_0, \dots, x_k) = \begin{cases}
      (x_0, \dots, \hat{x_i}, \dots, x_k) & \text{ if } l(x_0, \dots, \hat{x_i}, \dots, x_k) = l(x_0, \dots, x_k) \\
      0 & \text{otherwise}.
    \end{cases}
  \]
\end{definition}

\begin{remark}For a good tuple $(x_0, \dots, x_k)$,
  \[
    \partial_i(x_0, \dots, x_k) \neq 0 \iff d(x_{i-1}, x_i) + d(x_i, x_{i+1}) = d(x_{i-1}, x_{i+1}).
  \]  
\end{remark}

\begin{lemma}
  $\partial \circ \partial = 0$.
\end{lemma}

\begin{proof}
  aaa.
\end{proof}

\begin{definition}(\textit{magnitude homology})
  Let $G$ be a graph.
  The \textit{magnitude homology} $MH_{*,*}(G)$ of $G$ is the homology of the magnitude chain complex $MC_{*,*}(G)$, that is,
  \[
    MH_{k,l}(G) = \Ker \partial \cap (MC_{k,l}(G))  / \im \partial \cap (MC_{k,l}(G)).
  \]
\end{definition}

\begin{example}
  \begin{enumerate}[label=(\roman*)]
    \item $G = V_n$. Then,
    \[
      MC_{k,l}(V_n) = \begin{cases}
        \ZZ\{(x) | x \in V(V_n) \} & (k=l=0) \\
        0 & (\text{otherwise}).
      \end{cases}
    \]
    $\partial = 0$ inplies that
    \[
      MH_{k,l}(V_n) \approx \begin{cases}
        \ZZ^n & (k=l=0) \\
        0 & (\text{otherwise}).
      \end{cases}
    \]
    \item $G = K_n (n \geq 2)$. Then, $l(x_0, \cdots, x_k) = k$ for any good tuple $(x_0, \cdots, x_k)$ of $V(K_n)$. Thus,
    \[
      MC_{k,l}(K_n) = \begin{cases}
        \ZZ\{(x_0, \dots, x_k) | x_0 \neq x_1 \neq \cdots \neq x_k \} & (l=k) \\
        0 & (\text{otherwise}).
      \end{cases}
    \]
    $\partial = 0$ inplies that
    \[
      MH_{k,l}(K_n) \approx \begin{cases}
        \ZZ^{n(n-1)^l} & (l=k) \\
        0 & (\text{otherwise}).
      \end{cases}
    \]
    \item $G = C_5$. Number the vertices of $C_5$ as shown in the following figure. \\
    ここにナンバリングした$C_5$の図を挿入 \\
    Let us consider $MH_{2,3}(C_5)$.
    続く
  \end{enumerate}
\end{example}

\begin{theorem}\label{categorifying}
  Let $G$ be a graph. Then, 
  \[
    \sum_{k,l \geq 0} (-1)^k \rank (MH_{k,l}(G)) q^l = \# G ~ \text{in} ~  \ZZ[[q]].
  \] 
\end{theorem}

\begin{proof}
  \[
  \begin{split}
    (LHS) &= \sum_{l \geq 0} \chi(MH_{*,l}(G)) q^l \\
    &= \sum_{l \geq 0} \chi(MC_{*,l}(G))q^l \\
    &= \sum_{k,l \geq 0} (-1)^k \rank(MC_{k,l}(G))q^l \\
    &= \sum_{k \geq 0} (-1)^k \sum_{x_0 \neq \cdots \neq x_k} q^{d(x_0, x_1) + \cdots + d(x_{k-1}, x_k)} \\
    &= \# G.
  \end{split}
  \]
  The last equation is obtained by \cref{sum_of_magnitude}.
\end{proof}

\begin{proposition}
  Let $G$ be a graph. Then,
  \begin{itemize}
    \item $MH_{0,0}(G) \approx \ZZ^{|V(G)|}$.
    \item $MH_{1,1}(G) \approx \ZZ^{2|E(G)|}$.
  \end{itemize}
  holds.
\end{proposition}

\begin{proof}
  \[
      MC_{k,0}(G) = \begin{cases}
        \ZZ\{(x) | x \in V(G) \} & (k=0) \\
        0 & (\text{otherwise})
      \end{cases}
  \]
  and $\partial = 0$ induces the first equation.
  \[
      MC_{k,1}(G) = \begin{cases}
        \ZZ\{(x_0, x_1) | x_0 \neq x_1 \} & (k=1) \\
        0 & (\text{otherwise})
      \end{cases}
    \]
  and $\partial = 0$ induces the second equation.
\end{proof}

\begin{definition}
  The diameter $d$ of a graph $G$ is defined by
  \[
    d = \max \{ d(x,y) | x,y \in V(G) \text{ and } x,y \text{ lie in the same component of G}\}.
  \]
  If $G = V_n$, then we define $d=0$.
  Then, for any graph $G$, $0 \leq d < \infty$.

\end{definition}

\begin{proposition}
  Let $G$ be a graph and $d$ be the diameter of $G$ and assume that $MH_{k,l}(G) \neq 0$ for given non-negative integers $k$ and $l$.
  Then,
  \begin{itemize}
    \item $\frac{l}{d} \leq k \leq l$.
    \item If $d > 1 \text{ and } l > 0 \text{, then } \frac{l}{d} < k \leq l$.
  \end{itemize}
  holds.
\end{proposition}

\begin{proof}
  Since $MH_{k,l}(G) \neq 0$, there exists a good tuple $(x_0, \dots, x_k)$ of length $l$ such that $\partial (x_0, \dots, x_k) = 0$.
  Thus, $l = l(x_0, \dots, x_k) = \sum_{i=1}^{k} d(x_{i-1}, x_i) \leq \sum_{i=1}^{k} d = kd$ and $l = \sum_{i=1}^{k} d(x_{i-1}, x_i) \geq k$.
  This implies that $\frac{l}{d} \leq k \leq l$.
  
  Now, assume that $d > 1$ and $l > 0$ and suppose that $k = \frac{l}{d}$.
  From the above discussion, we have $d(x_i, x_{i+1}) = d$ for all $i$. $\partial(x_0, \cdots, x_k) = 0$. For the $(k+1)$-tuple $\partial(x_0, \cdots, x_k)$ is a linear combination of at most $k-1$ distinct terms of $k$-tuples, so $\partial(x_0, \cdots, x_k) = 0$ implies $d(x_{i-1}, x_i) + d(x_i, x_{i+1}) \neq d(x_{i-1}, x_{i+1})$ for all $1 \leq i \leq k-1$.
  Since $d(x_0, x_1) = d \geq 2$, there exists a vertex $y$ such that $d(x_0, y) + d(y, x_1) = d(x_0, x_1)$ and $y \neq x_0, x_1$.
  Then, $(x_0, y, x_1, \cdots, x_k)$ is a good tuple in $MC_{k+1, l}(G)$ and \[
    \partial_i(x_0, y, x_1, \cdots, x_k) =
    \begin{cases}
      (x_0, x_1, \cdots, x_k) & (i=1) \\
      0 & (2 \leq i \leq k).
    \end{cases}
  \].
  It is obvious for $3 \leq i$ by $d(x_{i-1}, x_i) + d(x_i, x_{i+1}) \neq d(x_{i-1}, x_{i+1})$ and is also true for $i=2$ since $d(y, x_1) + d(x_1, x_2) = d(y, x_1) + d > d \geq d(y, x_2)$. This implies $MH_{k,l}(G) = 0$ and contradicts the assumption.
\end{proof}

\subsection{Induced Maps}
\begin{definition}
  Let $G$ and $H$ be graphs.
  A map $f : V(G) \rightarrow V(H)$ is said to be a \textit{graph map} if for any $\{x,y\} \in E(G)$, either $f(x) = f(y)$ or $\{f(x), f(y)\} \in E(H)$.
\end{definition}

\begin{proposition}\label{f_inequality}
  $l(f(x_0), \cdots, f(x_k)) \leq l(x_0, \cdots, x_k)$ for any good tuple $(x_0, \cdots, x_k)$ of $V(G)$.
\end{proposition}

\begin{proof}
  For any vertices $x,y \in V(G)$, $d_H(f(x), f(y)) \leq d_G(x,y)$ holds. Indeed, if $x,y$ lie in the same component of $G$, then there exists a path $(x = x_0 \rightarrow x_1 \rightarrow \cdots \rightarrow x_n = y)$ in $G$ such that $n = d_G(x,y)$.
  Since $f$ is a graph map, either $f(x_{i-1}) = f(x_i)$ or $\{f(x_{i-1}), f(x_i)\} \in E(H)$ for any $1 \leq i \leq n$.
  Thus, $(f(x) = f(x_0) \rightarrow f(x_1) \rightarrow \cdots \rightarrow f(x_n) = f(y))$ is a path in $H$ and then $d_H(f(x), f(y)) \leq n = d_G(x,y)$.
  If $x,y$ do not lie in the same component, then $d_G(x,y) = d_H(f(x), f(y)) = \infty$. Then,
  \[
    \begin{split}
      l(f(x_0), \cdots, f(x_k)) &= \sum_{i=1}^{k} d_H(f(x_{i-1}), f(x_i)) \\
      &\leq \sum_{i=1}^{k} d_G(x_{i-1}, x_i) \\
      &= l(x_0, \cdots, x_k).
    \end{split}
  \]
\end{proof}

\begin{definition}
  Let $G$ and $H$ be graphs and $f : V(G) \rightarrow V(H)$ be a graph map.
  Then, the \textit{induced map} $f_{\#} : MC_{*,*}(G) \rightarrow MC_{*,*}(H)$ is defined by
  \[
    f_{\#}(x_0, \cdots, x_k) = \begin{cases}
      (f(x_0), \cdots, f(x_k)) & l(f(x_0), \cdots, f(x_k)) = l(x_0, \cdots, x_k) \\
      0 & \text{otherwise}
    \end{cases}
  \]
  for any good tuple $(x_0, \cdots, x_k)$ of $V(G)$.
\end{definition}

\begin{proposition}
  The induced map $f_{\#} : MC_{*,*}(G) \rightarrow MC_{*,*}(H)$ is a chain map.
\end{proposition}

\begin{proof}
  aaa.
\end{proof}

\begin{definition}(Induced maps in homoplogy)
  If $f : G \rightarrow H$ is a graph map, the \textit{induced map in homology} $f_* : MH_{*,*}(G) \rightarrow MH_{*,*}(H)$ is the map induced by the chain map $f_{\#} : MC_{*,*}(G) \rightarrow MC_{*,*}(H)$.
\end{definition}

\begin{proposition}
  The assignment $G \mapsto MH_{*,*}(G)$ and $f \mapsto f_*$ defines a functor from the category of graphs and graph maps to the category of bigraded abelian groups and bigraded homomorphisms, denoting by $\mathbf{Graph} \rightarrow \mathbf{BAb}$.
\end{proposition}

\begin{remark}
  $A$ is called a \textit{bigaded abelian group} if $A = \bigoplus_{k,l \geq 0} A_{k,l}$ where each $A_{k,l}$ is an abelian group.
  A \textit{bigraded homomorphism} $f : A \rightarrow B$ between bigraded abelian groups $A$ and $B$ is a homomorphism such that $f(A_{k,l}) \subset B_{k,l}$ for any $k,l \geq 0$.
\end{remark}

\begin{proposition}
  Let $f:G \rightarrow H$ be a graph map.
  \begin{itemize}
    \item $f_* : MH_{0,0}(G) \rightarrow MH_{0,0}(H)$ is given by $f_*(x) = f(x)$ for any $x \in V(G)$.
    \item $f_* : MH_{1,1}(G) \rightarrow MH_{1,1}(H)$ is given by
    \[
      f_*(x_0, x_1) = \begin{cases}
        (f(x_0), f(x_1)) & (\text{ if } ) \\
        0 & (\text{otherwise}).
      \end{cases}
    \]
    for any $(x_0, x_1) \in MH_{1,1}(G)$.
  \end{itemize}
\end{proposition}

\begin{proof}
  The first equation is obvious. \\
  For the second equation, we obtain by definition;
  \[
    f_*(x_0, x_1) = \begin{cases}
      (f(x_0), f(x_1)) & l(f(x_0), f(x_1)) = l(x_0, x_1) = 1 \\
      0 & \text{otherwise}
    \end{cases}
  \]
  for any $(x_0, x_1) \in MH_{1,1}(G)$. Since $f$ is a graph map, $l(f(x_0), f(x_1)) = 1$ if and only if $f(x_0) \neq f(x_1)$.
\end{proof}

\begin{corollary}
  Let $f : G \rightarrow H$ be a graph map.
  $f_*$ is an isomorphism if and only if $f$ is a graph isomorphism.
\end{corollary}

\begin{proof}
  aaa.
\end{proof}


\subsection{Disjoint Union}
\begin{proposition}\label{magnitude_of_disjoint_union}
  Let $G$ and $H$ be graphs.
  We define the inclusion graph maps $i : G \rightarrow G \bigsqcup H, j : H \rightarrow G \bigsqcup H$. Then,
  \[
    i_* \oplus j_* : MH_{*,*}(G) \oplus MH_{*,*}(H) \rightarrow MH_{*,*}(G \bigsqcup H)
  \]
  is an isomorphism for each $k,l \geq 0$.
\end{proposition}

\begin{proof}
  saaa.
\end{proof}

We obtain \cref{disjoint_union} by \cref{magnitude_of_disjoint_union} and $\chi(A_* \oplus B_*) = \chi(A_*) + \chi(B_*)$.




\subsection{Cartesian Products}
\begin{definition}This definition is not true.
  Fix $l \geq 0$. The \textit{exterior product} is the map
  \[
    \square : MC_{*,*}(G) \otimes MC_{*,*}(H) \rightarrow MC_{*,*}(G \square H)
  \]
  is defined as follows. Let $\square$ be the map
  \[
    \square : MC_{k_1, l_1}(G) \times MC_{k_2, l_2}(H) \rightarrow MC_{k, l}(G \square H) \text{ for } k_1, k_2 \geq 0, k = k_1 + k_2, l = l_1 + l_2,
  \]
  which is defined by
  \[
    \square((x_0, \cdots, x_{k_1}), (y_0, \cdots, y_{k_2})) = \sum_{\sigma} \sign(\sigma) ((x_{i_0}, y_{j_0}), (x_{i_1}, y_{j_1}), \cdots, (x_{i_k}, y_{j_k})),
  \]
  where the sum is over all shuffles $\sigma$ of type $(k_1, k_2)$, that is, all sequences
  \[    
    ((i_0, j_0), (i_1, j_1), \cdots, (i_k, j_k))
  \]
  such that
  \[
    i_0 = 0, j_0 = 0, 0 \leq i_r \leq k_1, 0 \leq j_r \leq k_2 \text{ for } 0 \leq r \leq k, \text{ and }
  \]

  \[
    (i_{r+1}, j_{r+1}) = \begin{cases}
      (i_r + 1, j_r) & \text{or} \\
      (i_r, j_r + 1) & \text{ for } 0 \leq r < k,
    \end{cases}
  \]
  and 
  \[
    \sign(\sigma) = (-1)^m,
  \]
  where $m = \# \{ (i,j) \in \{ \{0, \cdots, k_1\} \times \{0, \cdots, k_2\} \} | i = i_r \Rightarrow j < j_r \}$.

  Here, we extend the map $\square$ bilinearly to the tensor product \\
  \[
    MC_{k_1, l_1}(G) \otimes MC_{k_2, l_2}(H) \rightarrow MC_{k, l}(G \square H)
  \]. 
  
  We denote this induced map also by $\square$ and call it the \textit{exterior product}.
\end{definition}

\begin{example}
  Let $G = C_2 \square C_2 = C_4$ \\
  \graph{C_2_square.pdf} \\
  Consider the exterior product $\square((x_0, x_1) \otimes (y_0, y_1))$.
  We have the two shuffles of type $(1,1)$:
  \[
    ((0,0), (1,0), (1,1)), ~ ((0,0), (0,1), (1,1)).
  \]
  Thus,
  \[
    \begin{split}
      \square((x_0, x_1) \otimes (y_0, y_1)) &= - ((x_0, y_0), (x_1, y_0), (x_1, y_1)) + ((x_0, y_0), (x_0, y_1), (x_1, y_1)).
    \end{split}
  \]
\end{example}

\begin{remark}
  As you see in the above example, the number of shuffles is $\binom{k}{k_1}$.
\end{remark}

\begin{proposition}
  The exterior product $\square : MC_{*,*}(G) \otimes MC_{*,*}(H) \rightarrow MC_{*,*}(G \square H)$ is a chain map.
\end{proposition}

\begin{proof}
  Let $\bf{x} = (x_0, \cdots, x_{k_1}), \bf{y} = (y_0, \cdots, y_{k_2})$.
  Now, we show that 
  \[
    \partial \circ \square(\bf{x} \otimes \bf{y}) = \square((\partial \bf{x}) \otimes \bf{y}) + (-1)^{k_1} \square (\bf{x} \otimes (\partial \bf{y})) = \square \circ (\partial \otimes \partial) (\bf{x} \otimes \bf{y}).
  \]
  Here, we should consider the sequence of tensor products of the magnitude chain complexes defined by
  \begin{align*}
    \partial \otimes \partial : MC_{k_1, l_1}(G) \otimes MC_{k_2, l_2}(H) 
    &\rightarrow 
    MC_{k_1-1, l_1}(G) \otimes MC_{k_2, l_2}(H) \\
    &\quad \oplus MC_{k_1, l_1}(G) \otimes MC_{k_2-1, l_2}(H), \\
    (\partial \otimes \partial) (\mathbf{x} \otimes \mathbf{y}) 
    &= (\partial \mathbf{x}) \otimes \mathbf{y} + (-1)^{k_1} \mathbf{x} \otimes (\partial \mathbf{y}).
  \end{align*}
  Then, we should show only the first equality. \\
  ここに可換図式を挿入. 証明も続く
\end{proof}


From this proposition, we obtain the induced map in homology, also denoting $\square$.

\begin{definition}(Tor functor)
  Let $R$ be a ring and $A$ and $B$ be $R$-modules.
  Then, $\tor(A,B)$ is defined by the derived functor of the tensor product.  
\end{definition}

\begin{theorem}
  Let $G$ and $H$ be graphs.
  \[
    0 \rightarrow MH_{*,*}(G) \otimes MH_{*,*}(H) \xrightarrow{\square} MH_{*,*}(G \square H) \rightarrow \tor(MH_{*-1,*}(G), MH_{*,*}(H)) \rightarrow 0
  \]
  is a short exact sequence and non-naturally split.
  In particular, if $MH_{*,*}(G)$ or $MH_{*,*}(H)$ is torsion-free, then the exterior product $\square$ is an isomorphism.
\end{theorem}

We don't prove this theorem in this thesis.

\begin{example}
  $G = C_4 = C_2 \square C_2$.
\end{example}



\subsection{The Mayer-Vietoris Sequence}
\begin{definition}
  Let $X$ be a graph and $G,H$ be subgraphs of $X$.
  \begin{enumerate}
    \item $(X;G,H)$ is said to be a \textit{projecting decomposition} if $X = G \cup H$, $G \cap H$ is convex in $X$ and $H$ projects to $G \cap H$. \\ 
    We write $i^G : G \rightarrow X$, $i^H : H \rightarrow X$, $j^G : G \cap H \rightarrow G$, $j^H : G \cap H \rightarrow H$ for the inclusion graph maps.

    \item Let $(X;G,H), (X';G',H')$ be projecting decompositions. $f : (X;G,H) \rightarrow (X';G',H')$ is said to be a \textit{decomposition map} if $f : X \rightarrow X'$ is a graph map such that $f(V(G)) \subset V(G')$ and $f(V(H)) \subset V(H')$.
    \item Let $f : (X;G,H) \rightarrow (X';G',H')$ be a decomposition map.
    Then, $f$ is said to be a \textit{projecting decomposition map} if $V_{G \cap H}(H) = f^{-1}(V_{G' \cap H'}(H'))$ and $f(\pi(h)) = \pi(f(h))$ for any $h \in V_{G \cap H}(H)$.
    \item Let $(X;G,H)$ be a projecting decomposition. $MC_{*,*}(G,H)$ denote the subcomplex of $MC_{*,*}(X)$ spanned by good tuples $(x_0, \cdots, x_k)$ whose entries all lie in $G$ or all lie in $H$.
  \end{enumerate}
\end{definition}


\begin{theorem}
  Let $(X;G,H)$ be a projecting decomposition.
  Then, the inclusion map
  \[
    MC_{*,l}(G,H) \hookrightarrow MC_{*,l}(X)
  \]
  is a quasi-isomorphism for any $l \geq 0$.
\end{theorem}

\begin{proof}
  aaa.
\end{proof}

\begin{theorem}\label{Mayer-Vietoris}(the main theorem II)
  Let $(X;G,H)$ be a projecting decomposition.
  Then,
  \[
    0 \rightarrow MH_{*,*}(G \cap H) \xrightarrow{(j^G_*, -j^H_*)} MH_{*,*}(G) \oplus MH_{*,*}(H) \xrightarrow{i^G_* \oplus i^H_*} MH_{*,*}(X) \rightarrow 0
  \]
  is a split short exact sequence.
\end{theorem}

\begin{proof}
  aaa.
\end{proof}


\begin{corollary}
  Let $(X;G,H)$ be a projecting decomposition.
  Then,
  \[
    \# X = \# G + \# H - \# (G \cap H)
  \]
  in $\ZZ[[q]]$.
\end{corollary}

\begin{proof}
  By Theorem \ref{Mayer-Vietoris}, 
  \begin{align*}
    &\chi(MH_{*,l}(G \cap H)) - \chi((MH_{*,l}(G)) \oplus \chi(MH_{*,l}(H))) + \chi(MH_{*,l}(X)) = 0. \\
    \Rightarrow &\chi(MH_{*,l}(X)) = \chi(MH_{*,l}(G)) + \chi(MH_{*,l}(H)) - \chi(MH_{*,l}(G \cap H)).
  \end{align*}

  For each $l \geq 0$, multiplying by $q^l$ and summing over all $l \geq 0$, we have
  \[
    \sum_{l \geq 0} \chi(MH_{*,l}(X)) q^l = \sum_{l \geq 0} \chi(MH_{*,l}(G)) q^l + \sum_{l \geq 0} \chi(MH_{*,l}(H)) q^l - \sum_{l \geq 0} \chi(MH_{*,l}(G \cap H)) q^l.
  \]
  By \cref{categorifying}, we obtain the desired equation.
\end{proof}

\begin{corollary}
  Let $T$ be a tree.
\end{corollary}


\subsection{Diagonal Graphs}
\begin{definition}
  A graph $G$ is said to be \textit{diagonal} if $MH_{k,l}(G) = 0$ for $k \neq l$.
\end{definition}

\begin{lemma}
  A tree is diagonal.
\end{lemma}

\begin{proof}
  aaa
\end{proof}

\begin{proposition}
  For a diagonal graph, the magnitude completely determines the magnitude homology ranks.
\end{proposition}

\begin{proof}
  Obvious by \cref{categorifying}.
\end{proof}



% Section 4
\newpage
\section{Motivation: The Magnitude of Enriched Categories}
In this section, we explain the motivation for studying the magnitude of graphs in a broader context. We employ the notion of enriched categories to define the magnitude.

\subsection{The Magnitude of a Matrix}
\begin{definition}
  Let $k$ be a set and $+, \cdot $ be a binary operation on $k$, and $0_k, 1_k$ be elements of $k$.
  Then, $(k, +, \cdot, 0_k, 1_k)$ is called a \textit{rig} if the following conditions hold:
  \begin{itemize}
    \item $(k, +, 0_k)$ is a commutative monoid.
    \item $(k, \cdot, 1_k)$ is a monoid.
    \item multiplication distributes over addition.
  \end{itemize}
  Now, we mean a rig as a commutative rig with the operation $\cdot$.
\end{definition}

\begin{example}
  $(\ZZ_{\geq 0}, +, \cdot, 0, 1)$ is a rig.
\end{example}

\begin{definition}
  Let $k$ be a rig and $I, J$ be finite sets.
  A $I \times J$-matrix is a function $\zeta : I \times J \rightarrow k$.
\end{definition}

\begin{remark}
  Let $k$ be a rig, and $I, J, L$ be finite sets.
  \begin{enumerate}
    \item If $\zeta_1$ is an $I \times J$-matrix and $\zeta_2$ is a $J \times L$-matrix, then the product $\zeta_1 \zeta_2$ is defined as follows:
    \[
      (\zeta_1 \zeta_2)(i,l) = \sum_{j \in J} \zeta_1(i,j) \cdot \zeta_2(j,l) \quad (i \in I, l \in L)
    \]

    \item $\delta : I \times I \rightarrow k$ is called the \textit{identity matrix} if $\delta(i,j) = 1_k$ when $i=j$ and $\delta(i,j) = 0_k$ when $i \neq j$.
    
    \item Let $\zeta : I \times J \rightarrow k$ be a matrix. We define $\zeta^* : J \times I \rightarrow k$ by $\zeta^*(j,i) = \zeta(i,j)$.
    
    \item Let $\zeta$ be an $I \times I$-matrix. If there exists an $I \times I$-matrix $\zeta^{-1}$ such that $\zeta \zeta^{-1} = \delta$ and $\zeta^{-1} \zeta = \delta$, then $\zeta$ is said to be \textit{invertible} and $\zeta^{-1}$ is called the \textit{inverse} of $\zeta$.
    
    \item $w : I \rightarrow k$ is called a \textit{vector}. $w$ can be thought of as an element of $k^I$. If $\zeta$ is an $I \times J$-matrix, $v$ is a $I$-vector, and $w$ is a $J$-vector, then the product $\zeta w : I \rightarrow k$ and $v \zeta : J \rightarrow k$ are defined by
    \[
      (\zeta w)(i) = \sum_{j \in J} \zeta(i,j) \cdot w(j) \quad (i \in I)
    \]
    \[
      (v \zeta)(j) = \sum_{i \in I} v(i) \cdot \zeta(i,j) \quad (j \in J)
    \]
    Now, $\zeta w$ is a $I$-vector and $v \zeta$ is a $J$-vector.

    \item If $w, v$ are $I$-vectors, then the \textit{inner product} $v w$ is defined by
    \[
      v w = \sum_{i \in I} v(i) \cdot w(i)
    \]

    \item A vector $u_I : I \rightarrow k$ is defined by $u_I(i) = 1_k$ for any $i \in I$.
  \end{enumerate}
\end{remark}


\begin{definition}
  Let $\zeta$ be an $I \times I$-matrix over a rig $k$.
  \begin{itemize}
    \item A \textit{weighting} on $\zeta$ is a vector $w : J \rightarrow k$ such that $\zeta w = u_I$.
    $w(j)$ is called the \textit{weight} of $j \in J$.
    \item A \textit{coweighting} on $\zeta$ is a vector $v : I \rightarrow k$ such that $v \zeta = u_I^*$. $v(i)$ is called the \textit{coweight} of $i \in I$.
  \end{itemize}
\end{definition}

\begin{example}
  Let $G$ be a graph.
  Then, $Z_G(q)$ is a $V(G) \times V(G)$-matrix over the rig $\QQ [[ q]]$ and the weighting on $Z_G(q)$ is the weighting on $G$ defined in Section 2.1.
\end{example}

\begin{lemma}\label{sum_of_w.c.}
  Let $\zeta$ be an $I \times I$-matrix over a rig $k$.
  If $\zeta$ has a weighting $w$ and a coweighting $v$, then
  \[
    \sum_{i \in I} v(i) = \sum_{j \in J} w(j)
  \]
\end{lemma}

\begin{proof}
  \[
    \begin{split}
      \sum_{i \in I} v(i) &= v u_I \\
      &= v (\zeta w) \\
      &= (v \zeta) w \\
      &= u_J w \\
      &= \sum_{j \in J} w(j)
    \end{split}
  \]
\end{proof}

From this lemma, the sum of the weighting or coweighting on $\zeta$ is unique if they exist. Therefore, we can define the magnitude of $\zeta$ as follows:

\begin{definition}
  Let $\zeta$ be an $I \times J$-matrix over a rig $k$.
  If $\zeta$ has a weighting and a coweighting, then the \textit{magnitude} of $\zeta$ is defined to be
  \[
    \# \zeta = \sum_{i \in I} v(i) = \sum_{j \in J} w(j)
  \]
  where $w$ is the weighting on $\zeta$ and $v$ is the coweighting on $\zeta$.
\end{definition}

\begin{lemma}
  Let $\zeta$ be an $I \times I$-matrix over a rig $k$.
  \begin{enumerate}
    \item If $\zeta$ is invertible, then $\zeta$ has the magnitude.
    \item If $k$ is a field and $\zeta$ has the magnitude, then $\zeta$ is invertible.
  \end{enumerate}
\end{lemma}

\begin{proof}
  (1) If $\zeta$ is invertible, then $w = \zeta^{-1} u_I$ and $v = u_I \zeta^{-1}$ obviously satisfy the definition of weighting and coweighting respectively. Thus $\zeta$ has the magnitude by \cref{sum_of_w.c.}.

  (2) If $k$ is a field and $\zeta$ has the magnitude, then there exist a weighting $w$ and a coweighting $v$ on $\zeta$.
  Let $\zeta x$ be a zero-map for some $x : I \rightarrow k$. Then,
  \[
    0 = v (\zeta x) = (v \zeta) x = u_I x = \sum_{i \in I} x(i)
  \]
  ここからやり直し
\end{proof}

\begin{lemma}
  Let $\zeta$ be an invertible $I \times I$-matrix over a rig $k$.
  Then, $\zeta$ has the unique weighting $w$ of $\zeta$, given by $w(j) = \sum_{i \in I} \zeta^{-1}(j,i)$ for $j \in I$, and the unique coweighting $v$ of $\zeta$, given by $v(i) = \sum_{j \in I} \zeta^{-1}(j,i)$ for $i \in I$.
  Then, \[
    \# \zeta = \sum_{i,j \in I} \zeta^{-1}(j,i)
  \]
\end{lemma}

\begin{proof}
  We should check the uniqueness and it holds from the invertibility of $\zeta$.
\end{proof}


\subsection{The Definition of Enriched Categories}
In this document, we only treat the locally small categories, which means that for any objects $a,b$ of a category $\CC$, the hom-set $\hom_{\CC}(a,b)$ is a set.



\begin{definition}
  A category $\CC$ is called a \textit{monoid} if $\CC$ has only one object $*$ and $V=Hom_{\CC}(*,*)$ is a monoid with the composition of morphisms as the binary operation and the identity morphism $id_*$ as the identity element. We denote the operation of $V$ by $\otimes$.
  ここに可換図式を挿入
\end{definition}

\begin{definition}
  A pair $(\VV, \otimes, I)$ is called a \textit{monoidal category} if it satisfies the following conditions:
  \begin{enumerate}
    \item $\VV$ is a category.
    \item $\otimes : \VV \times \VV \rightarrow \VV$ is a functor.
    \item $I$ is an object of $\VV$.
    \item There exist the natural isomorphism $\alpha : \otimes \circ (\otimes \times id_{\VV}) \Rightarrow \otimes \circ (id_{\VV} \times \otimes)$ given by $\alpha_{uvw} : (u \otimes v) \otimes w \isom u \otimes (v \otimes w)$.
    \item There exist the natural isomorphism $\lambda : I \otimes - \Rightarrow id_{\VV}$ given by $\lambda_u : I \otimes u \isom u$.
    \item There exist the natural isomorphism $\rho : - \otimes I \Rightarrow id_{\VV}$ given by $\rho_u : u \otimes I \isom u$.
    \item The following diagram commutes for any objects $u,v,w,x$ of $\VV$:
    ここに可換図式を挿入
    \item The following diagrams commute for any objects $u,v$ of $\VV$:
    ここに可換図式を挿入
  \end{enumerate}
\end{definition}

\begin{example}
  \begin{enumerate}[label=(\roman*)]
    \item $(\textit{Set}, \times, \{*\})$ is a monoidal category.
    ここに説明を挿入
    \item $(\textit{Vect}_K, \otimes_K, K)$ is a monoidal category, where $K$ is a field.
    ここに説明を挿入
    \item $([0, \infty], +, 0)$ is a monoidal category.
    ここに説明を挿入
    \item $(\mathbf{2}, \otimes, t)$ is a monoidal category, where $\mathbf{2}$ is the category defined by $Ob(\mathbf{2}) = \{t,f\}$ and the morphism sets are defined by
    \begin{center}
      \begin{tabular}{c|cc}
        $\hom_{\mathbf{2}}$ & $t$ & $f$ \\
        \hline
        $t$ & $\{id_t\}$ & $\emptyset$ \\
        $f$ & $\{*\}$ & $\{id_f\}$ \\
      \end{tabular}
    \end{center}
    and the operation $\otimes$ is defined by the following table:
    \begin{center}
      \begin{tabular}{c|cc}
        $\otimes$ & $t$ & $f$ \\
        \hline
        $t$ & $t$ & $f$ \\
        $f$ & $f$ & $f$ \\
      \end{tabular}
    \end{center}
    ここに説明を挿入

    Then, $\mathbf{2}$ is a monoidal subcategory of $[0, \infty]$ by the embedding $t \mapsto 0, f \mapsto \infty$ and of $\textit{Set}$ by the embedding $t \mapsto \{*\}, f \mapsto \emptyset$.
  \end{enumerate}
\end{example}

\begin{definition}
  An \textit{enriched category} $\calA$ in a monoidal category $(\VV, \otimes, I)$ is defined as follows:

  \begin{enumerate}
    \item For any objects $a,b$ of $\calA$, $\hom_{\calA}(a,b)$ is an object of $\VV$.
    \item For any objects $a,b,c$ of $\calA$, there exists a morphism $m_{abc} : \hom_{\calA}(b,c) \otimes \hom_{\calA}(a,b) \rightarrow \hom_{\calA}(a,c)$ in $\VV$, which defines the composition of morphisms.
    \item For any object $a$ of $\calA$, there exists a morphism $j_a : I \rightarrow \hom_{\calA}(a,a)$ in $\VV$, which defines the identity morphism of $a$.
    あと3つの可換図式を挿入
  \end{enumerate}

  Then, $\calA$ is called a \textit{$\VV$-category}.
\end{definition}

\begin{definition}
  Let $\calA, \calA'$ be $\VV$-categories. A functor $F:\calA \rightarrow \calA'$ is called a \textit{$\VV$-functor} if it satisfies the following conditions. We denote $F_{ab} : \mor_{\calA}(a,b) \rightarrow \mor_{\calA'}(F(a),F(b))$ as the morphism function.
  \begin{enumerate}
    \item The following diagram commutes for any objects $a,b,c$ of $\calA$:
    ここに可換図式を挿入
    \item The following diagram commutes for any object $a$ of $\calA$:
    ここに可換図式を挿入
  \end{enumerate}
\end{definition}

\begin{remark}
  The family of all $\VV$-categories and $\VV$-functors form a category, which is denoted by $\VV\textit{-Cat}$.
\end{remark}

\begin{example}
  aaa
\end{example}


\subsection{The Magnitude of Enriched Categories}
\begin{definition}
  Let $(\VV, \otimes, I)$ be a monoidal category and $k$ be a rig.
  We define a monoid homomorphism
  \[  | \cdot | : (Ob(\VV)/\approx, \otimes, I) \rightarrow (k, \cdot, 1_k)
  \]
  such that $|I| = 1_k$ and $|u \otimes v| = |u| \cdot |v|$ for any objects $u,v$ of $\VV$.
\end{definition}


\begin{example}
  aaa.
\end{example}

\begin{definition}
  Let $\calA$ be a $\VV$-category.
\end{definition}




  \subsection{The Relation of The Magnitudes of Graphs and Enriched Categories}
ここが私が一番説明したい部分です.






% Appendix
\newpage
\appendix
\counterwithin{theorem}{section}
\addcontentsline{toc}{section}{Appendix}
\section*{Appendix}

\section{The calculation of Graph Automorphisms}
\begin{proposition}
  Let $G = K_{n,n}$. Then, $\aut(G) \approx (\mathfrak{S}_n \times \mathfrak{S}_n) \rtimes_s \ZZ_2$, where $s : \ZZ_2 \rightarrow \aut(G); 0 \mapsto id_G, 1 \mapsto \tau$ and $\tau$ is the automorphism which interchanges the two parts of $K_{n,n}$.
\end{proposition}

\begin{proof}
  Now, 
  \[
    0 \rightarrow \mathfrak{S}_n \times \mathfrak{S}_n \xrightarrow{incl} \aut(G) \xrightarrow{s} \ZZ_2 \rightarrow 0
  \]
  is exact and this sequence splits. Thus, we have $\aut(G) \approx (\mathfrak{S}_n \times \mathfrak{S}_n) \rtimes_s \ZZ_2$.
\end{proof}





% Bibliography (参考文献)
\newpage
\begin{thebibliography}{9}

\bibitem{Lamport94}
Leslie Lamport.
\textit{LaTeX: A Document Preparation System}.
Addison-Wesley, 2nd edition, 1994.

\bibitem{Knuth84}
Donald E. Knuth.
\textit{The TeXbook}.
Addison-Wesley, 1984.

\end{thebibliography}



\end{document}
